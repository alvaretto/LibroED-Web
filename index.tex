% Options for packages loaded elsewhere
\PassOptionsToPackage{unicode}{hyperref}
\PassOptionsToPackage{hyphens}{url}
\PassOptionsToPackage{dvipsnames,svgnames,x11names}{xcolor}
%
\documentclass[
  letterpaper,
]{book}

\usepackage{amsmath,amssymb}
\usepackage{iftex}
\ifPDFTeX
  \usepackage[T1]{fontenc}
  \usepackage[utf8]{inputenc}
  \usepackage{textcomp} % provide euro and other symbols
\else % if luatex or xetex
  \usepackage{unicode-math}
  \defaultfontfeatures{Scale=MatchLowercase}
  \defaultfontfeatures[\rmfamily]{Ligatures=TeX,Scale=1}
\fi
\usepackage{lmodern}
\ifPDFTeX\else  
    % xetex/luatex font selection
\fi
% Use upquote if available, for straight quotes in verbatim environments
\IfFileExists{upquote.sty}{\usepackage{upquote}}{}
\IfFileExists{microtype.sty}{% use microtype if available
  \usepackage[]{microtype}
  \UseMicrotypeSet[protrusion]{basicmath} % disable protrusion for tt fonts
}{}
\makeatletter
\@ifundefined{KOMAClassName}{% if non-KOMA class
  \IfFileExists{parskip.sty}{%
    \usepackage{parskip}
  }{% else
    \setlength{\parindent}{0pt}
    \setlength{\parskip}{6pt plus 2pt minus 1pt}}
}{% if KOMA class
  \KOMAoptions{parskip=half}}
\makeatother
\usepackage{xcolor}
\setlength{\emergencystretch}{3em} % prevent overfull lines
\setcounter{secnumdepth}{5}
% Make \paragraph and \subparagraph free-standing
\ifx\paragraph\undefined\else
  \let\oldparagraph\paragraph
  \renewcommand{\paragraph}[1]{\oldparagraph{#1}\mbox{}}
\fi
\ifx\subparagraph\undefined\else
  \let\oldsubparagraph\subparagraph
  \renewcommand{\subparagraph}[1]{\oldsubparagraph{#1}\mbox{}}
\fi

\usepackage{color}
\usepackage{fancyvrb}
\newcommand{\VerbBar}{|}
\newcommand{\VERB}{\Verb[commandchars=\\\{\}]}
\DefineVerbatimEnvironment{Highlighting}{Verbatim}{commandchars=\\\{\}}
% Add ',fontsize=\small' for more characters per line
\usepackage{framed}
\definecolor{shadecolor}{RGB}{241,243,245}
\newenvironment{Shaded}{\begin{snugshade}}{\end{snugshade}}
\newcommand{\AlertTok}[1]{\textcolor[rgb]{0.68,0.00,0.00}{#1}}
\newcommand{\AnnotationTok}[1]{\textcolor[rgb]{0.37,0.37,0.37}{#1}}
\newcommand{\AttributeTok}[1]{\textcolor[rgb]{0.40,0.45,0.13}{#1}}
\newcommand{\BaseNTok}[1]{\textcolor[rgb]{0.68,0.00,0.00}{#1}}
\newcommand{\BuiltInTok}[1]{\textcolor[rgb]{0.00,0.23,0.31}{#1}}
\newcommand{\CharTok}[1]{\textcolor[rgb]{0.13,0.47,0.30}{#1}}
\newcommand{\CommentTok}[1]{\textcolor[rgb]{0.37,0.37,0.37}{#1}}
\newcommand{\CommentVarTok}[1]{\textcolor[rgb]{0.37,0.37,0.37}{\textit{#1}}}
\newcommand{\ConstantTok}[1]{\textcolor[rgb]{0.56,0.35,0.01}{#1}}
\newcommand{\ControlFlowTok}[1]{\textcolor[rgb]{0.00,0.23,0.31}{#1}}
\newcommand{\DataTypeTok}[1]{\textcolor[rgb]{0.68,0.00,0.00}{#1}}
\newcommand{\DecValTok}[1]{\textcolor[rgb]{0.68,0.00,0.00}{#1}}
\newcommand{\DocumentationTok}[1]{\textcolor[rgb]{0.37,0.37,0.37}{\textit{#1}}}
\newcommand{\ErrorTok}[1]{\textcolor[rgb]{0.68,0.00,0.00}{#1}}
\newcommand{\ExtensionTok}[1]{\textcolor[rgb]{0.00,0.23,0.31}{#1}}
\newcommand{\FloatTok}[1]{\textcolor[rgb]{0.68,0.00,0.00}{#1}}
\newcommand{\FunctionTok}[1]{\textcolor[rgb]{0.28,0.35,0.67}{#1}}
\newcommand{\ImportTok}[1]{\textcolor[rgb]{0.00,0.46,0.62}{#1}}
\newcommand{\InformationTok}[1]{\textcolor[rgb]{0.37,0.37,0.37}{#1}}
\newcommand{\KeywordTok}[1]{\textcolor[rgb]{0.00,0.23,0.31}{#1}}
\newcommand{\NormalTok}[1]{\textcolor[rgb]{0.00,0.23,0.31}{#1}}
\newcommand{\OperatorTok}[1]{\textcolor[rgb]{0.37,0.37,0.37}{#1}}
\newcommand{\OtherTok}[1]{\textcolor[rgb]{0.00,0.23,0.31}{#1}}
\newcommand{\PreprocessorTok}[1]{\textcolor[rgb]{0.68,0.00,0.00}{#1}}
\newcommand{\RegionMarkerTok}[1]{\textcolor[rgb]{0.00,0.23,0.31}{#1}}
\newcommand{\SpecialCharTok}[1]{\textcolor[rgb]{0.37,0.37,0.37}{#1}}
\newcommand{\SpecialStringTok}[1]{\textcolor[rgb]{0.13,0.47,0.30}{#1}}
\newcommand{\StringTok}[1]{\textcolor[rgb]{0.13,0.47,0.30}{#1}}
\newcommand{\VariableTok}[1]{\textcolor[rgb]{0.07,0.07,0.07}{#1}}
\newcommand{\VerbatimStringTok}[1]{\textcolor[rgb]{0.13,0.47,0.30}{#1}}
\newcommand{\WarningTok}[1]{\textcolor[rgb]{0.37,0.37,0.37}{\textit{#1}}}

\providecommand{\tightlist}{%
  \setlength{\itemsep}{0pt}\setlength{\parskip}{0pt}}\usepackage{longtable,booktabs,array}
\usepackage{calc} % for calculating minipage widths
% Correct order of tables after \paragraph or \subparagraph
\usepackage{etoolbox}
\makeatletter
\patchcmd\longtable{\par}{\if@noskipsec\mbox{}\fi\par}{}{}
\makeatother
% Allow footnotes in longtable head/foot
\IfFileExists{footnotehyper.sty}{\usepackage{footnotehyper}}{\usepackage{footnote}}
\makesavenoteenv{longtable}
\usepackage{graphicx}
\makeatletter
\def\maxwidth{\ifdim\Gin@nat@width>\linewidth\linewidth\else\Gin@nat@width\fi}
\def\maxheight{\ifdim\Gin@nat@height>\textheight\textheight\else\Gin@nat@height\fi}
\makeatother
% Scale images if necessary, so that they will not overflow the page
% margins by default, and it is still possible to overwrite the defaults
% using explicit options in \includegraphics[width, height, ...]{}
\setkeys{Gin}{width=\maxwidth,height=\maxheight,keepaspectratio}
% Set default figure placement to htbp
\makeatletter
\def\fps@figure{htbp}
\makeatother

\makeatletter
\@ifpackageloaded{tcolorbox}{}{\usepackage[skins,breakable]{tcolorbox}}
\@ifpackageloaded{fontawesome5}{}{\usepackage{fontawesome5}}
\definecolor{quarto-callout-color}{HTML}{909090}
\definecolor{quarto-callout-note-color}{HTML}{0758E5}
\definecolor{quarto-callout-important-color}{HTML}{CC1914}
\definecolor{quarto-callout-warning-color}{HTML}{EB9113}
\definecolor{quarto-callout-tip-color}{HTML}{00A047}
\definecolor{quarto-callout-caution-color}{HTML}{FC5300}
\definecolor{quarto-callout-color-frame}{HTML}{acacac}
\definecolor{quarto-callout-note-color-frame}{HTML}{4582ec}
\definecolor{quarto-callout-important-color-frame}{HTML}{d9534f}
\definecolor{quarto-callout-warning-color-frame}{HTML}{f0ad4e}
\definecolor{quarto-callout-tip-color-frame}{HTML}{02b875}
\definecolor{quarto-callout-caution-color-frame}{HTML}{fd7e14}
\makeatother
\makeatletter
\@ifpackageloaded{bookmark}{}{\usepackage{bookmark}}
\makeatother
\makeatletter
\@ifpackageloaded{caption}{}{\usepackage{caption}}
\AtBeginDocument{%
\ifdefined\contentsname
  \renewcommand*\contentsname{Table of contents}
\else
  \newcommand\contentsname{Table of contents}
\fi
\ifdefined\listfigurename
  \renewcommand*\listfigurename{List of Figures}
\else
  \newcommand\listfigurename{List of Figures}
\fi
\ifdefined\listtablename
  \renewcommand*\listtablename{List of Tables}
\else
  \newcommand\listtablename{List of Tables}
\fi
\ifdefined\figurename
  \renewcommand*\figurename{Figure}
\else
  \newcommand\figurename{Figure}
\fi
\ifdefined\tablename
  \renewcommand*\tablename{Table}
\else
  \newcommand\tablename{Table}
\fi
}
\@ifpackageloaded{float}{}{\usepackage{float}}
\floatstyle{ruled}
\@ifundefined{c@chapter}{\newfloat{codelisting}{h}{lop}}{\newfloat{codelisting}{h}{lop}[chapter]}
\floatname{codelisting}{Listing}
\newcommand*\listoflistings{\listof{codelisting}{List of Listings}}
\makeatother
\makeatletter
\makeatother
\makeatletter
\@ifpackageloaded{caption}{}{\usepackage{caption}}
\@ifpackageloaded{subcaption}{}{\usepackage{subcaption}}
\makeatother
\ifLuaTeX
  \usepackage{selnolig}  % disable illegal ligatures
\fi
\usepackage{bookmark}

\IfFileExists{xurl.sty}{\usepackage{xurl}}{} % add URL line breaks if available
\urlstyle{same} % disable monospaced font for URLs
\hypersetup{
  pdftitle={Métodos Clásicos de Resolución de E.D.O.},
  pdfauthor={Material Educativo},
  colorlinks=true,
  linkcolor={Maroon},
  filecolor={Maroon},
  citecolor={Blue},
  urlcolor={Blue},
  pdfcreator={LaTeX via pandoc}}

\title{Métodos Clásicos de Resolución de E.D.O.}
\usepackage{etoolbox}
\makeatletter
\providecommand{\subtitle}[1]{% add subtitle to \maketitle
  \apptocmd{\@title}{\par {\large #1 \par}}{}{}
}
\makeatother
\subtitle{Ecuaciones Diferenciales Ordinarias}
\author{Material Educativo}
\date{2026-02-11}

\begin{document}
\frontmatter
\maketitle

\renewcommand*\contentsname{Table of contents}
{
\hypersetup{linkcolor=}
\setcounter{tocdepth}{2}
\tableofcontents
}
\mainmatter
\bookmarksetup{startatroot}

\chapter{Métodos Clásicos de Resolución de
E.D.O.}\label{muxe9todos-cluxe1sicos-de-resoluciuxf3n-de-e.d.o.}

Ecuaciones Diferenciales Ordinarias

\hfill\break

\bookmarksetup{startatroot}

\chapter*{Bienvenida}\label{bienvenida}
\addcontentsline{toc}{chapter}{Bienvenida}

\markboth{Bienvenida}{Bienvenida}

Este libro presenta los \textbf{métodos clásicos de resolución de
ecuaciones diferenciales ordinarias (E.D.O.)}, organizados de manera
sistemática y pedagógica.

\section{Contenido del libro}\label{contenido-del-libro}

El material está estructurado en \textbf{tres partes principales}:

\subsection{Parte I: Ecuaciones Explícitas de Primer
Orden}\label{parte-i-ecuaciones-expluxedcitas-de-primer-orden}

Métodos fundamentales para resolver ecuaciones del tipo \(y' = f(x,y)\):

\begin{itemize}
\tightlist
\item
  Variables separadas
\item
  Ecuaciones homogéneas
\item
  Ecuaciones lineales
\item
  Ecuación de Bernoulli
\item
  Ecuaciones exactas
\item
  Factor integrante
\end{itemize}

\subsection{Parte II: E.D. con Derivada
Implícita}\label{parte-ii-e.d.-con-derivada-impluxedcita}

Técnicas especiales para ecuaciones donde la derivada no está despejada:

\begin{itemize}
\tightlist
\item
  Ecuación de Clairaut
\item
  Ecuación de Lagrange
\item
  Ecuación de d'Alembert
\end{itemize}

\subsection{Parte III: Reducción de
Orden}\label{parte-iii-reducciuxf3n-de-orden}

Métodos avanzados para ecuaciones de orden superior:

\begin{itemize}
\tightlist
\item
  Reducción en ausencia de variable dependiente
\item
  Reducción en ausencia de variable independiente
\item
  Ecuaciones homogéneas generalizadas
\end{itemize}

\section{Características del
material}\label{caracteruxedsticas-del-material}

\begin{tcolorbox}[enhanced jigsaw, breakable, colframe=quarto-callout-tip-color-frame, left=2mm, coltitle=black, opacityback=0, colbacktitle=quarto-callout-tip-color!10!white, bottomtitle=1mm, title=\textcolor{quarto-callout-tip-color}{\faLightbulb}\hspace{0.5em}{RECETAS}, titlerule=0mm, arc=.35mm, bottomrule=.15mm, toptitle=1mm, colback=white, rightrule=.15mm, toprule=.15mm, opacitybacktitle=0.6, leftrule=.75mm]

A lo largo del libro encontrarás \textbf{recetas} que resumen paso a
paso los métodos de resolución.

\end{tcolorbox}

\begin{tcolorbox}[enhanced jigsaw, breakable, colframe=quarto-callout-note-color-frame, left=2mm, coltitle=black, opacityback=0, colbacktitle=quarto-callout-note-color!10!white, bottomtitle=1mm, title=\textcolor{quarto-callout-note-color}{\faInfo}\hspace{0.5em}{Ejemplos resueltos}, titlerule=0mm, arc=.35mm, bottomrule=.15mm, toptitle=1mm, colback=white, rightrule=.15mm, toprule=.15mm, opacitybacktitle=0.6, leftrule=.75mm]

Cada método incluye ejemplos completamente desarrollados con
explicaciones detalladas.

\end{tcolorbox}

\begin{tcolorbox}[enhanced jigsaw, breakable, colframe=quarto-callout-warning-color-frame, left=2mm, coltitle=black, opacityback=0, colbacktitle=quarto-callout-warning-color!10!white, bottomtitle=1mm, title=\textcolor{quarto-callout-warning-color}{\faExclamationTriangle}\hspace{0.5em}{Ejercicios}, titlerule=0mm, arc=.35mm, bottomrule=.15mm, toptitle=1mm, colback=white, rightrule=.15mm, toprule=.15mm, opacitybacktitle=0.6, leftrule=.75mm]

Al final de cada apartado hay ejercicios propuestos para practicar.

\end{tcolorbox}

\section{Navegación}\label{navegaciuxf3n}

Utiliza el menú lateral para navegar entre los diferentes apartados. La
barra de búsqueda te permitirá encontrar rápidamente conceptos
específicos.

\begin{center}\rule{0.5\linewidth}{0.5pt}\end{center}

\textbf{Material Educativo} • 2026

\part{Parte I: Ecuaciones Explícitas de Primer Orden}

\chapter{Apartado 1: Variables
Separadas}\label{apartado-1-variables-separadas}

\section{Variables separadas}\label{variables-separadas}

Si tenemos la E. D. \[ g(x) = h(y)y', \] formalmente, podemos poner
\(g(x) \, dx = h(y) \, dy\); si suponemos que \(G\) es una primitiva de
\(g\) y \(H\) una de \(h\), tendremos \(G'(x) \, dx = H'(y) \, dy\) e,
integrando, \(G(x) = H(y) + C\), que es la solución general de la
ecuación.

Expliquemos con un poco más de rigor por qué funciona el método: Sea
\(y = \varphi(x)\) una solución de la E. D., es decir, \(\varphi(x)\)
debe cumplir \(g(x) = h(\varphi(x))\varphi'(x)\). Pero \(H\) es una
primitiva de \(h\), así que, por la regla de la cadena,
\(g(x) = h(\varphi(x))\varphi'(x) = (H \circ \varphi)'(x)\). Integrando,
\(G(x) = (H \circ \varphi)(x) + C\) (lo que antes hemos expresado como
\(G(x) = H(y) + C\)), de donde \(\varphi(x) = H^{-1}(G(x) - C)\).

En los pasos anteriores, está justificado emplear la regla de la cadena
cuando \(\varphi\) y \(H\) son derivables, lo cual es cierto sin más que
suponer que \(h\) sea continua. Y finalmente, para poder despejar
\(\varphi\) mediante el uso de \(H^{-1}\) bastaría con exigir además que
\(h\) no se anulara en el intervalo de definición, con lo cual, como
\(H' = h \neq 0\), \(H\) es creciente o decreciente luego existe
\(H^{-1}\) (en otras palabras, como la derivada de \(H\) no se anula, el
teorema de la función inversa nos asegura que existe \(H^{-1}\)).

Las ecuaciones en variables separadas son las más sencillas de integrar
y, a la vez, las más importantes, ya que cualquier otro método de
resolución se basa esencialmente en aplicar diversos trucos para llegar
a una ecuación en variables separadas. En ellas hemos visto, con todo
rigor, qué hipótesis hay que imponer para que el método que conduce a la
solución esté correctamente empleado, y cómo se justifica el
funcionamiento del proceso. A partir de ahora no incidiremos más en
estos detalles que, aunque importantes, sobrecargarían la explicación.
El lector puede detenerse mentalmente a pensar en ellos, justificando
adecuadamente los pasos que se efectúen.

En cualquier caso, conviene recordar que la expresión \(\frac{dy}{dx}\)
es simplemente una útil notación para designar la derivada de \(y\)
respecto de \(x\), no un cociente de \(dy\) dividido por \(dx\); ni
\(dy\) ni \(dx\) tienen entidad en sí mismas. Esta notación se emplea,
no porque sí ni para introducir confusión, sino que, al contrario, se
usa porque es consecuente con los enunciados de varios importantes
resultados. Ya hemos visto cómo resulta adecuada a la hora de recordar
cómo resolver ecuaciones en variables separadas
\(g(x) = h(y)\frac{dy}{dx}\), descomponiendo \(g(x) \, dx = h(x) \, dy\)
(como si \(\frac{dy}{dx}\) fuese realmente una fracción) e integrando
ambos lados de la expresión anterior. Pero no sólo aquí se pone de
manifiesto la utilidad de esta notación. Por ejemplo, el teorema de la
función inversa prueba (con las hipótesis adecuadas) que, cuando \(y\)
es una función de \(x\), si se despeja \(x\) como función de \(y\) se
cumple \[ x'(y) = \frac{dx}{dy} = \frac{1}{dy/dx} = \frac{1}{y'(x)}, \]
es decir, se produce un comportamiento similar a si estuviéramos
operando con fracciones. Análogamente, si tenemos que \(z\) es una
función de \(y\) y, a su vez, \(y\) una función de \(x\), la regla de la
cadena establece que la derivada de la función compuesta \(z(x)\) es
\[ \frac{dz}{dx} = \frac{dz}{dy} \frac{dy}{dx}, \] que es como si
simplificáramos \(dy\) en los supuestos cocientes de la derecha. Esto
permite usar las notaciones del tipo \(\frac{dy}{dx}\) y su
comportamiento como si fuesen fracciones como regla nemotécnica de los
resultados anteriores.

\begin{tcolorbox}[enhanced jigsaw, breakable, colframe=quarto-callout-tip-color-frame, left=2mm, coltitle=black, opacityback=0, colbacktitle=quarto-callout-tip-color!10!white, bottomtitle=1mm, title=\textcolor{quarto-callout-tip-color}{\faLightbulb}\hspace{0.5em}{RECETA 1. Variables separadas}, titlerule=0mm, arc=.35mm, bottomrule=.15mm, toptitle=1mm, colback=white, rightrule=.15mm, toprule=.15mm, opacitybacktitle=0.6, leftrule=.75mm]

Son de la forma \[ g(x) = h(y)y'. \] Formalmente, se separa
\(g(x) = h(y)\frac{dy}{dx}\) en \(g(x) \, dx = h(y) \, dy\) y se
integra.

\end{tcolorbox}

\subsection*{Ejemplo 1}\label{ejemplo-1}
\addcontentsline{toc}{subsection}{Ejemplo 1}

\textbf{Resolver} \[ \frac{dy}{dx} + (\operatorname{sen} x)y = 0. \]

\textbf{Solución:}

Despejando, \(\frac{dy}{y} = -(\operatorname{sen} x) \, dx\) e,
integrando, \(\log y = \cos x + C\), es decir, \(y = e^{\cos x + C}\).
Sin más que tomar \(K = e^C\) encontramos las soluciones
\(y = K e^{\cos x}\).

Fijarse que, en principio, parece que \(K\) tiene que ser positiva; pero
en realidad la integral de \(\frac{dy}{y}\) es \(\log |y|\), lo que nos
llevaría a soluciones con valores negativos de \(K\). Por último, notar
\(y = 0\) (es decir, tomar \(K = 0\)) también es claramente una solución
de la E. D., aunque no se obtiene con el método seguido.

\begin{tcolorbox}[enhanced jigsaw, breakable, colframe=quarto-callout-important-color-frame, left=2mm, coltitle=black, opacityback=0, colbacktitle=quarto-callout-important-color!10!white, bottomtitle=1mm, title=\textcolor{quarto-callout-important-color}{\faExclamation}\hspace{0.5em}{Important}, titlerule=0mm, arc=.35mm, bottomrule=.15mm, toptitle=1mm, colback=white, rightrule=.15mm, toprule=.15mm, opacitybacktitle=0.6, leftrule=.75mm]

Así pues, la solución general de la E. D. es de la forma
\(y = K e^{\cos x}\) con \(K \in \mathbb{R}\).

\end{tcolorbox}

\begin{center}\rule{0.5\linewidth}{0.5pt}\end{center}

\subsection*{Visualización
interactiva}\label{visualizaciuxf3n-interactiva}
\addcontentsline{toc}{subsection}{Visualización interactiva}

Puedes explorar cómo cambia la familia de soluciones con diferentes
valores de \(K\):

\begin{Shaded}
\begin{Highlighting}[]
\NormalTok{//| echo: false}
\NormalTok{viewof K = Inputs.range([{-}3, 3], \{}
\NormalTok{  value: 1,}
\NormalTok{  step: 0.1,}
\NormalTok{  label: "Constante K:"}
\NormalTok{\})}

\NormalTok{Plot.plot(\{}
\NormalTok{  marks: [}
\NormalTok{    Plot.line(d3.range({-}6, 6, 0.1).map(x =\textgreater{} (\{}
\NormalTok{      x: x,}
\NormalTok{      y: K * Math.exp(Math.cos(x))}
\NormalTok{    \})), \{x: "x", y: "y", stroke: "blue", strokeWidth: 2\}),}
\NormalTok{    Plot.ruleX([0], \{stroke: "gray"\}),}
\NormalTok{    Plot.ruleY([0], \{stroke: "gray"\})}
\NormalTok{  ],}
\NormalTok{  grid: true,}
\NormalTok{  x: \{domain: [{-}6, 6], label: "x"\},}
\NormalTok{  y: \{domain: [{-}5, 5], label: "y"\},}
\NormalTok{  width: 700,}
\NormalTok{  height: 400,}
\NormalTok{  style: \{}
\NormalTok{    fontSize: "12px"}
\NormalTok{  \}}
\NormalTok{\})}
\end{Highlighting}
\end{Shaded}

\begin{tcolorbox}[enhanced jigsaw, breakable, colframe=quarto-callout-note-color-frame, left=2mm, coltitle=black, opacityback=0, colbacktitle=quarto-callout-note-color!10!white, bottomtitle=1mm, title=\textcolor{quarto-callout-note-color}{\faInfo}\hspace{0.5em}{Observación}, titlerule=0mm, arc=.35mm, bottomrule=.15mm, toptitle=1mm, colback=white, rightrule=.15mm, toprule=.15mm, opacitybacktitle=0.6, leftrule=.75mm]

Nota cómo todas las curvas pasan por puntos donde \(\cos x\) alcanza sus
máximos y mínimos.

\end{tcolorbox}

\chapter{\texorpdfstring{Apartado 2: Ecuación de la forma
\(y' = f(ax + by)\)}{Apartado 2: Ecuación de la forma y\textquotesingle{} = f(ax + by)}}\label{apartado-2-ecuaciuxf3n-de-la-forma-y-fax-by}

\section{\texorpdfstring{Ecuación de la forma
\(y' = f(ax + by)\)}{Ecuación de la forma y\textquotesingle{} = f(ax + by)}}\label{ecuaciuxf3n-de-la-forma-y-fax-by}

Si \(a = 0\) o \(b = 0\), la ecuación es separable. En otro caso,
efectuemos el cambio de función \(y(x)\) por \(z(x)\) dado por
\(z = ax + by\), de donde \(z' = a + by'\), por tanto,
\(y' = \frac{z'-a}{b}\). Entonces, sustituyendo en la E. D. obtenemos
\(\frac{z'-a}{b} = f(z)\), es decir, \(z' = a + bf(z)\), que es de
variables separadas. La escribimos como \[ dx = \frac{dz}{a + bf(z)}, \]
con lo que, integrando,
\(x = \int (a + bf(z))^{-1} \, dz = \phi(z, C)\). Así pues, las
soluciones de la E. D. de partida serán \[ x = \phi(ax + by, C), \] de
modo que hemos encontrado \(y\) como función de \(x\) expresada en forma
implícita.

\begin{tcolorbox}[enhanced jigsaw, breakable, colframe=quarto-callout-tip-color-frame, left=2mm, coltitle=black, opacityback=0, colbacktitle=quarto-callout-tip-color!10!white, bottomtitle=1mm, title=\textcolor{quarto-callout-tip-color}{\faLightbulb}\hspace{0.5em}{RECETA 2. Ecuación de la forma \(y' = f(ax + by)\)}, titlerule=0mm, arc=.35mm, bottomrule=.15mm, toptitle=1mm, colback=white, rightrule=.15mm, toprule=.15mm, opacitybacktitle=0.6, leftrule=.75mm]

El cambio de función \(y(x)\) por \(z(x)\) dado por \(z = ax + by\) la
transforma en una de variables separadas.

\end{tcolorbox}

\subsection*{Ejemplo 2}\label{ejemplo-2}
\addcontentsline{toc}{subsection}{Ejemplo 2}

\textbf{Resolver} \[ y' - e^x e^y = -1. \]

\textbf{Solución:}

Tenemos \(y' + 1 = e^{x+y}\), con lo que si efectuamos el cambio de
función dado por la sustitución \(z = x + y\), la ecuación queda
transformada en \(z' = e^z\), es decir, \(dx = e^{-z} \, dz\), ecuación
en variables separadas cuya solución es \(x = -e^{-z} + C\).

Volviendo a las variables iniciales, \(C - x = e^{-x-y}\), de donde
\(\log(C - x) = -x - y\), y por tanto la solución de la E. D. de partida
es \(y = -\log(C - x) - x\).

\begin{tcolorbox}[enhanced jigsaw, breakable, colframe=quarto-callout-note-color-frame, left=2mm, coltitle=black, opacityback=0, colbacktitle=quarto-callout-note-color!10!white, bottomtitle=1mm, title=\textcolor{quarto-callout-note-color}{\faInfo}\hspace{0.5em}{Observación}, titlerule=0mm, arc=.35mm, bottomrule=.15mm, toptitle=1mm, colback=white, rightrule=.15mm, toprule=.15mm, opacitybacktitle=0.6, leftrule=.75mm]

Observar que no nos hemos preocupado de poner módulos cuando al calcular
una integral aparece un logaritmo. El lector podría analizar estos casos
con mucho más cuidado.

\end{tcolorbox}

\chapter{Apartado 3: Ecuaciones
Homogéneas}\label{apartado-3-ecuaciones-homoguxe9neas}

\section{Ecuaciones Homogéneas}\label{ecuaciones-homoguxe9neas}

Supongamos que tenemos la ecuación \[ y' = f\left(\frac{y}{x}\right). \]

Para resolverla, hacemos el cambio de función \(y(x)\) por \(u(x)\)
mediante \(u = \frac{y}{x}\). Así, derivando \(y = ux\) tenemos
\(y' = u'x + u\), es decir, \(u'x + u = f(u)\). Esta ecuación, que
podemos poner como \(u'x = f(u) - u\), es de variables separadas. Vamos
a solucionarla:

\subsection*{\texorpdfstring{Caso 1:
\(f(u) \neq u\)}{Caso 1: f(u) \textbackslash neq u}}\label{caso-1-fu-neq-u}
\addcontentsline{toc}{subsection}{Caso 1: \(f(u) \neq u\)}

Podemos escribir \(\frac{du}{f(u)-u} = \frac{dx}{x}\) e, integrando,
\(\int \frac{du}{f(u)-u} = \log(\frac{x}{C})\). Despejando \(x\)
obtenemos \(x = C e^{\phi(u)}\) con
\(\phi(u) = \int \frac{du}{f(u)-u}\).

Por tanto, las curvas con ecuaciones paramétricas \[
\begin{cases}
x = C e^{\phi(u)} \\
y = C u e^{\phi(u)}
\end{cases}
\] son solución de la ecuación diferencial para cada
\(C \in \mathbb{R}\).

\begin{tcolorbox}[enhanced jigsaw, breakable, colframe=quarto-callout-note-color-frame, left=2mm, coltitle=black, opacityback=0, colbacktitle=quarto-callout-note-color!10!white, bottomtitle=1mm, title=\textcolor{quarto-callout-note-color}{\faInfo}\hspace{0.5em}{Familia de curvas homotéticas}, titlerule=0mm, arc=.35mm, bottomrule=.15mm, toptitle=1mm, colback=white, rightrule=.15mm, toprule=.15mm, opacitybacktitle=0.6, leftrule=.75mm]

Esto constituye una familia de curvas homotéticas: una curva se obtiene
de otra mediante una homotecia, es decir, multiplicando los valores de
\(x\) e \(y\) por una constante.

\end{tcolorbox}

A veces, es conveniente expresar estas soluciones de otras formas.
Siempre puede ponerse \(x = C e^{\phi(y/x)}\), solución dada mediante
una función implícita. Y, cuando en \(x = C e^{\phi(u)}\) se logra
despejar de alguna forma \(u = H(x, C)\), la solución de la E. D. queda
mucho más sencilla: \(y = xH(x, C)\).

\subsection*{Caso 2: Soluciones
singulares}\label{caso-2-soluciones-singulares}
\addcontentsline{toc}{subsection}{Caso 2: Soluciones singulares}

Supongamos ahora que existe algún \(u_0\) tal que \(f(u_0) = u_0\). En
este caso, es inmediato comprobar que la recta \(y = u_0x\) es solución:
\(y' = u_0 = f(u_0) = f(\frac{y}{x})\), luego se satisface la ecuación
diferencial.

\begin{tcolorbox}[enhanced jigsaw, breakable, colframe=quarto-callout-important-color-frame, left=2mm, coltitle=black, opacityback=0, colbacktitle=quarto-callout-important-color!10!white, bottomtitle=1mm, title=\textcolor{quarto-callout-important-color}{\faExclamation}\hspace{0.5em}{Soluciones singulares}, titlerule=0mm, arc=.35mm, bottomrule=.15mm, toptitle=1mm, colback=white, rightrule=.15mm, toprule=.15mm, opacitybacktitle=0.6, leftrule=.75mm]

Este tipo de soluciones que no se obtienen con el procedimiento general
suelen denominarse \textbf{soluciones singulares}.

\end{tcolorbox}

\subsection*{Nota sobre funciones
homogéneas}\label{nota-sobre-funciones-homoguxe9neas}
\addcontentsline{toc}{subsection}{Nota sobre funciones homogéneas}

En general, una función \(h(x, y)\) se dice \textbf{homogénea de grado
\(\alpha\)} si \(h(\lambda x, \lambda y) = \lambda^\alpha h(x, y)\).

Es inmediato comprobar que una E. D. de la forma
\[ P(x, y) \, dx + Q(x, y) \, dy = 0 \] con \(P(x, y)\) y \(Q(x, y)\)
funciones homogéneas del mismo grado es, efectivamente, una ecuación
diferencial homogénea (despejar
\(y' = \frac{dy}{dx} = -\frac{P(x,y)}{Q(x,y)} = -\frac{P(x,x(y/x))}{Q(x,x(y/x))}\)
y extraer \(\lambda = x\) de \(P\) y \(Q\)). De aquí proviene el nombre
de este tipo de ecuaciones.

\begin{tcolorbox}[enhanced jigsaw, breakable, colframe=quarto-callout-tip-color-frame, left=2mm, coltitle=black, opacityback=0, colbacktitle=quarto-callout-tip-color!10!white, bottomtitle=1mm, title=\textcolor{quarto-callout-tip-color}{\faLightbulb}\hspace{0.5em}{RECETA 3. Homogéneas}, titlerule=0mm, arc=.35mm, bottomrule=.15mm, toptitle=1mm, colback=white, rightrule=.15mm, toprule=.15mm, opacitybacktitle=0.6, leftrule=.75mm]

Son de la forma \[ y' = f\left(\frac{y}{x}\right). \]

Se hace el cambio de función \(y(x)\) por \(u(x)\) mediante \(y = ux\),
transformándose así la E. D. en una de variables separadas.

\end{tcolorbox}

\subsection*{Ejemplo 3}\label{ejemplo-3}
\addcontentsline{toc}{subsection}{Ejemplo 3}

\textbf{Resolver} \[ y' = \frac{2xy - y^2}{x^2}. \]

\textbf{Solución:}

Con el cambio \(y = ux\) podemos poner
\(y' = 2\frac{y}{x} - (\frac{y}{x})^2 = 2u - u^2\). Como
\(y' = u'x + u\), sustituyendo tenemos \(u'x + u = 2u - u^2\), es decir,
\(xu' = u - u^2\).

Si \(u \neq u^2\), podemos poner \(\frac{du}{u-u^2} = \frac{dx}{x}\).
Para integrar, descomponemos
\(\frac{1}{u-u^2} = \frac{A}{u} + \frac{B}{1-u}\), lo que se satisface
para \(A = B = 1\).

Entonces, integrando, \(\log u - \log(1 - u) = \log \frac{x}{C}\), es
decir, \(\frac{u}{1-u} = \frac{x}{C}\); y sustituyendo
\(u = \frac{y}{x}\) tenemos \(\frac{y/x}{1-y/x} = \frac{x}{C}\), de
donde \(Cy = x(x - y)\). De aquí es fácil despejar explícitamente \(y\)
si así se desea.

Por otra parte, a partir de \(u_0 = 0\) y \(u_0 = 1\) (para las cuales
\(u = u^2\)), se tienen las soluciones singulares \(y = 0\) e \(y = x\).

\chapter{Apartado 4: Ecuaciones
Exactas}\label{apartado-4-ecuaciones-exactas}

\section{Ecuaciones Exactas}\label{ecuaciones-exactas}

Llamamos exacta a una ecuación diferencial
\[ P(x, y) \, dx + Q(x, y) \, dy = 0, \] es decir,
\(y' = \frac{dy}{dx} = -\frac{P(x,y)}{Q(x,y)}\), que cumple
\(P_y = Q_x\) (con la notación \(P_y = \frac{\partial P}{\partial y}\),
\(Q_x = \frac{\partial Q}{\partial x}\)).

\subsection*{Expresiones diferenciales}\label{expresiones-diferenciales}
\addcontentsline{toc}{subsection}{Expresiones diferenciales}

Una expresión diferencial \(P(x, y) \, dx + Q(x, y) \, dy\) se dice que
es una \textbf{diferencial cerrada} en una región \(R\) del plano \(xy\)
si se verifica \(P_y(x, y) = Q_x(x, y)\) para todo \((x, y) \in R\).

Y se dice \textbf{exacta} en \(R\) cuando existe alguna función
\(F(x, y)\) tal que \(\frac{\partial F}{\partial x} = P\) y
\(\frac{\partial F}{\partial y} = Q\) para todo \((x, y) \in R\); en
otras palabras, si la diferencial de \(F\) es \(dF = P \, dx + Q \, dy\)
(\(F\), que es única salvo constantes, se denomina \textbf{función
potencial}).

\begin{tcolorbox}[enhanced jigsaw, breakable, colframe=quarto-callout-note-color-frame, left=2mm, coltitle=black, opacityback=0, colbacktitle=quarto-callout-note-color!10!white, bottomtitle=1mm, title=\textcolor{quarto-callout-note-color}{\faInfo}\hspace{0.5em}{Teorema de Schwartz}, titlerule=0mm, arc=.35mm, bottomrule=.15mm, toptitle=1mm, colback=white, rightrule=.15mm, toprule=.15mm, opacitybacktitle=0.6, leftrule=.75mm]

El teorema de Schwartz sobre igualdad de derivadas cruzadas nos asegura
que cualquier expresión diferencial exacta es cerrada. Lo contrario no
es cierto en general, aunque sí en dominios \textbf{simplemente conexos}
(sin agujeros).

\end{tcolorbox}

\subsection*{Método de resolución}\label{muxe9todo-de-resoluciuxf3n}
\addcontentsline{toc}{subsection}{Método de resolución}

Si tenemos \(P \, dx + Q \, dy = 0\) exacta, como existe \(F\) tal que
\(dF = P \, dx + Q \, dy\), entonces la ecuación podemos ponerla en la
forma \(dF = 0\) y, por tanto, su solución será \(F(x, y) = C\) (siendo
\(C\) constante arbitraria). Así pues, basta con que encontremos la
función potencial \(F\).

El procedimiento para hallarla es:

\begin{enumerate}
\def\labelenumi{\arabic{enumi}.}
\item
  Buscamos \(F\) tal que \(\frac{\partial F}{\partial x} = P\);
  integrando \(P(x, y)\) respecto a \(x\) mientras se mantiene \(y\)
  constante: \[F(x, y) = \int P(x, y) \, dx + \varphi(y)\] donde
  \(\varphi(y)\) es una función arbitraria.
\item
  Derivando respecto de \(y\):
  \[\frac{\partial F}{\partial y} = \frac{\partial}{\partial y} \int P(x, y) \, dx + \varphi'(y)\]
\item
  Como \(\frac{\partial F}{\partial y} = Q\), resulta:
  \[\varphi'(y) = Q(x, y) - \frac{\partial}{\partial y} \int P(x, y) \, dx\]
\item
  Integrando obtenemos \(\varphi(y)\) y, sustituyendo su valor, llegamos
  a \(F(x, y)\).
\item
  Las soluciones quedan expresadas implícitamente como \(F(x, y) = C\).
\end{enumerate}

\begin{tcolorbox}[enhanced jigsaw, breakable, colframe=quarto-callout-tip-color-frame, left=2mm, coltitle=black, opacityback=0, colbacktitle=quarto-callout-tip-color!10!white, bottomtitle=1mm, title=\textcolor{quarto-callout-tip-color}{\faLightbulb}\hspace{0.5em}{RECETA 4. Ecuaciones exactas}, titlerule=0mm, arc=.35mm, bottomrule=.15mm, toptitle=1mm, colback=white, rightrule=.15mm, toprule=.15mm, opacitybacktitle=0.6, leftrule=.75mm]

Son las de la forma \[ P(x, y) \, dx + Q(x, y) \, dy = 0, \] que cumplen
\(P_y = Q_x\). Se busca una función \(F(x, y)\) tal que
\(dF = P \, dx + Q \, dy\), y la solución de la E. D. es
\(F(x, y) = C\).

\end{tcolorbox}

\subsection*{Ejemplo 4}\label{ejemplo-4}
\addcontentsline{toc}{subsection}{Ejemplo 4}

\textbf{Resolver} \[ 3y + e^x + (3x + \cos y)y' = 0. \]

\textbf{Solución:}

Ponemos la ecuación en la forma \(P \, dx + Q \, dy = 0\) con: -
\(P(x, y) = 3y + e^x\) - \(Q(x, y) = 3x + \cos y\)

Verificamos: \(P_y = 3 = Q_x\), luego la E. D. es exacta.

Calculemos la función potencial \(F\):

\begin{enumerate}
\def\labelenumi{\arabic{enumi}.}
\item
  Como \(F_x = 3y + e^x\), integrando respecto de \(x\):
  \[F(x, y) = 3yx + e^x + \varphi(y)\]
\item
  Derivando respecto de \(y\) e igualando a \(Q\):
  \[3x + \varphi'(y) = 3x + \cos y\]
\item
  Por tanto \(\varphi'(y) = \cos y\), de donde
  \(\varphi(y) = \operatorname{sen} y\).
\item
  Así: \(F(x, y) = 3yx + e^x + \operatorname{sen} y\)
\end{enumerate}

\begin{tcolorbox}[enhanced jigsaw, breakable, colframe=quarto-callout-important-color-frame, left=2mm, coltitle=black, opacityback=0, colbacktitle=quarto-callout-important-color!10!white, bottomtitle=1mm, title=\textcolor{quarto-callout-important-color}{\faExclamation}\hspace{0.5em}{Solución}, titlerule=0mm, arc=.35mm, bottomrule=.15mm, toptitle=1mm, colback=white, rightrule=.15mm, toprule=.15mm, opacitybacktitle=0.6, leftrule=.75mm]

La solución de la E. D. viene dada, implícitamente, por:
\[3yx + e^x + \operatorname{sen} y = C\]

\end{tcolorbox}

\chapter{Apartado 5: Ecuaciones
Lineales}\label{apartado-5-ecuaciones-lineales}

\section{Ecuaciones Lineales de Primer
Orden}\label{ecuaciones-lineales-de-primer-orden}

Una ecuación diferencial lineal de primer orden es de la forma:
\[ y' + a(x)y = b(x) \]

donde \(a(x)\) y \(b(x)\) son funciones dadas de \(x\).

\subsection*{Método del Factor
Integrante}\label{muxe9todo-del-factor-integrante}
\addcontentsline{toc}{subsection}{Método del Factor Integrante}

El método de resolución consiste en multiplicar ambos lados de la
ecuación por un \textbf{factor integrante} \(\mu(x)\) que convierte el
lado izquierdo en la derivada de un producto.

\begin{tcolorbox}[enhanced jigsaw, breakable, colframe=quarto-callout-tip-color-frame, left=2mm, coltitle=black, opacityback=0, colbacktitle=quarto-callout-tip-color!10!white, bottomtitle=1mm, title=\textcolor{quarto-callout-tip-color}{\faLightbulb}\hspace{0.5em}{RECETA 5. Ecuaciones Lineales}, titlerule=0mm, arc=.35mm, bottomrule=.15mm, toptitle=1mm, colback=white, rightrule=.15mm, toprule=.15mm, opacitybacktitle=0.6, leftrule=.75mm]

Para resolver \(y' + a(x)y = b(x)\):

\begin{enumerate}
\def\labelenumi{\arabic{enumi}.}
\item
  Calcular el factor integrante: \(\mu(x) = e^{\int a(x) \, dx}\)
\item
  Multiplicar la ecuación por \(\mu(x)\):
  \[\mu(x)y' + \mu(x)a(x)y = \mu(x)b(x)\]
\item
  El lado izquierdo es \(\frac{d}{dx}[\mu(x)y]\):
  \[\frac{d}{dx}[\mu(x)y] = \mu(x)b(x)\]
\item
  Integrar ambos lados: \[\mu(x)y = \int \mu(x)b(x) \, dx + C\]
\item
  Despejar \(y\):
  \[y = \frac{1}{\mu(x)}\left[\int \mu(x)b(x) \, dx + C\right]\]
\end{enumerate}

\end{tcolorbox}

\subsection*{Ejemplo 5}\label{ejemplo-5}
\addcontentsline{toc}{subsection}{Ejemplo 5}

\textbf{Resolver} \[ y' + 2xy = x \]

\textbf{Solución:}

\begin{enumerate}
\def\labelenumi{\arabic{enumi}.}
\item
  Identificamos: \(a(x) = 2x\), \(b(x) = x\)
\item
  Factor integrante: \[\mu(x) = e^{\int 2x \, dx} = e^{x^2}\]
\item
  Multiplicamos: \[e^{x^2}y' + 2xe^{x^2}y = xe^{x^2}\]
\item
  El lado izquierdo es la derivada:
  \[\frac{d}{dx}[e^{x^2}y] = xe^{x^2}\]
\item
  Integramos:
  \[e^{x^2}y = \int xe^{x^2} \, dx = \frac{1}{2}e^{x^2} + C\]
\item
  Solución: \[y = \frac{1}{2} + Ce^{-x^2}\]
\end{enumerate}

\begin{tcolorbox}[enhanced jigsaw, breakable, colframe=quarto-callout-note-color-frame, left=2mm, coltitle=black, opacityback=0, colbacktitle=quarto-callout-note-color!10!white, bottomtitle=1mm, title=\textcolor{quarto-callout-note-color}{\faInfo}\hspace{0.5em}{Observación}, titlerule=0mm, arc=.35mm, bottomrule=.15mm, toptitle=1mm, colback=white, rightrule=.15mm, toprule=.15mm, opacitybacktitle=0.6, leftrule=.75mm]

La solución general es la suma de: - \textbf{Solución particular} de la
ecuación completa: \(y_p = \frac{1}{2}\) - \textbf{Solución general} de
la ecuación homogénea: \(y_h = Ce^{-x^2}\)

\end{tcolorbox}

\chapter{Apartado 6: Ecuación de
Bernoulli}\label{apartado-6-ecuaciuxf3n-de-bernoulli}

\section{Ecuación de Bernoulli}\label{ecuaciuxf3n-de-bernoulli}

La ecuación de Bernoulli es de la forma: \[ y' + a(x)y = b(x)y^n \]

donde \(n\) es un número real (\(n \neq 0, 1\)).

\begin{tcolorbox}[enhanced jigsaw, breakable, colframe=quarto-callout-note-color-frame, left=2mm, coltitle=black, opacityback=0, colbacktitle=quarto-callout-note-color!10!white, bottomtitle=1mm, title=\textcolor{quarto-callout-note-color}{\faInfo}\hspace{0.5em}{Casos especiales}, titlerule=0mm, arc=.35mm, bottomrule=.15mm, toptitle=1mm, colback=white, rightrule=.15mm, toprule=.15mm, opacitybacktitle=0.6, leftrule=.75mm]

\begin{itemize}
\tightlist
\item
  Si \(n = 0\): la ecuación es \textbf{lineal}
\item
  Si \(n = 1\): la ecuación es de \textbf{variables separadas}
\end{itemize}

\end{tcolorbox}

\subsection*{Método de resolución}\label{muxe9todo-de-resoluciuxf3n-1}
\addcontentsline{toc}{subsection}{Método de resolución}

El método consiste en hacer un cambio de variable que transforma la
ecuación de Bernoulli en una ecuación lineal.

\textbf{Cambio de variable:} \(z = y^{1-n}\)

Derivando: \(z' = (1-n)y^{-n}y'\), es decir, \(y' = \frac{y^n z'}{1-n}\)

Sustituyendo en la ecuación original:
\[\frac{y^n z'}{1-n} + a(x)y = b(x)y^n\]

Dividiendo por \(y^n\): \[\frac{z'}{1-n} + a(x)y^{1-n} = b(x)\]

Como \(z = y^{1-n}\): \[\frac{z'}{1-n} + a(x)z = b(x)\]

Multiplicando por \((1-n)\): \[z' + (1-n)a(x)z = (1-n)b(x)\]

Esta es una \textbf{ecuación lineal} en \(z\) que podemos resolver con
el método del factor integrante.

\begin{tcolorbox}[enhanced jigsaw, breakable, colframe=quarto-callout-tip-color-frame, left=2mm, coltitle=black, opacityback=0, colbacktitle=quarto-callout-tip-color!10!white, bottomtitle=1mm, title=\textcolor{quarto-callout-tip-color}{\faLightbulb}\hspace{0.5em}{RECETA 6. Ecuación de Bernoulli}, titlerule=0mm, arc=.35mm, bottomrule=.15mm, toptitle=1mm, colback=white, rightrule=.15mm, toprule=.15mm, opacitybacktitle=0.6, leftrule=.75mm]

Para resolver \(y' + a(x)y = b(x)y^n\) con \(n \neq 0, 1\):

\begin{enumerate}
\def\labelenumi{\arabic{enumi}.}
\item
  Hacer el cambio \(z = y^{1-n}\)
\item
  La ecuación se transforma en la lineal:
  \[z' + (1-n)a(x)z = (1-n)b(x)\]
\item
  Resolver la ecuación lineal en \(z\)
\item
  Deshacer el cambio: \(y = z^{\frac{1}{1-n}}\)
\end{enumerate}

\end{tcolorbox}

\subsection*{Ejemplo 6}\label{ejemplo-6}
\addcontentsline{toc}{subsection}{Ejemplo 6}

\textbf{Resolver} \[ y' + \frac{y}{x} = y^2 \]

\textbf{Solución:}

Esta es una ecuación de Bernoulli con \(a(x) = \frac{1}{x}\),
\(b(x) = 1\), \(n = 2\).

\begin{enumerate}
\def\labelenumi{\arabic{enumi}.}
\item
  Cambio de variable: \(z = y^{1-2} = y^{-1} = \frac{1}{y}\)
\item
  Derivando: \(z' = -\frac{y'}{y^2}\), por tanto \(y' = -y^2 z'\)
\item
  Sustituyendo en la ecuación: \[-y^2 z' + \frac{y}{x} = y^2\]
\item
  Dividiendo por \(y^2\): \[-z' + \frac{1}{xy} = 1\]
\item
  Como \(z = \frac{1}{y}\): \[-z' + \frac{z}{x} = 1\]
\item
  Reordenando: \[z' - \frac{z}{x} = -1\]
\item
  Esta es una ecuación lineal. Factor integrante:
  \[\mu(x) = e^{\int -\frac{1}{x} \, dx} = e^{-\ln x} = \frac{1}{x}\]
\item
  Multiplicamos: \[\frac{z'}{x} - \frac{z}{x^2} = -\frac{1}{x}\]
\item
  Integrando: \[\frac{z}{x} = -\ln|x| + C\]
\item
  Despejando \(z\): \[z = -x\ln|x| + Cx\]
\item
  Deshaciendo el cambio \(y = \frac{1}{z}\):
  \[y = \frac{1}{Cx - x\ln|x|}\]
\end{enumerate}

\begin{tcolorbox}[enhanced jigsaw, breakable, colframe=quarto-callout-important-color-frame, left=2mm, coltitle=black, opacityback=0, colbacktitle=quarto-callout-important-color!10!white, bottomtitle=1mm, title=\textcolor{quarto-callout-important-color}{\faExclamation}\hspace{0.5em}{Solución general}, titlerule=0mm, arc=.35mm, bottomrule=.15mm, toptitle=1mm, colback=white, rightrule=.15mm, toprule=.15mm, opacitybacktitle=0.6, leftrule=.75mm]

\[y = \frac{1}{x(C - \ln|x|)}\]

\end{tcolorbox}

\part{Parte II: E.D. con Derivada Implícita}

\chapter{\texorpdfstring{Apartado 7: \(F\) algebraica en \(y'\) de grado
\(n\)}{Apartado 7: F algebraica en y\textquotesingle{} de grado n}}\label{apartado-7-f-algebraica-en-y-de-grado-n}

\section{\texorpdfstring{Ecuaciones algebraicas en
\(y'\)}{Ecuaciones algebraicas en y\textquotesingle{}}}\label{ecuaciones-algebraicas-en-y}

Tenemos
\[ (y')^n + a_1(x, y)(y')^{n-1} + \dots + a_{n-1}(x, y)y' + a_n(x, y) = 0. \]

Resolviendo este polinomio en \(y'\) de grado \(n\) igualado a cero
obtenemos
\[ (y' - f_1(x, y))(y' - f_2(x, y))\cdots(y' - f_n(x, y)) = 0. \]

Por lo tanto, las soluciones de la E. D. de partida serán las de cada
una de las nuevas ecuaciones diferenciales \(y' - f_i(x, y) = 0\),
\(i = 1, 2, \dots, n\), que habrá que resolver. De esta forma obtenemos
\(n\) familias uniparamétricas de soluciones.

\begin{tcolorbox}[enhanced jigsaw, breakable, colframe=quarto-callout-tip-color-frame, left=2mm, coltitle=black, opacityback=0, colbacktitle=quarto-callout-tip-color!10!white, bottomtitle=1mm, title=\textcolor{quarto-callout-tip-color}{\faLightbulb}\hspace{0.5em}{RECETA 7. \(F\) algebraica en \(y'\) de grado \(n\)}, titlerule=0mm, arc=.35mm, bottomrule=.15mm, toptitle=1mm, colback=white, rightrule=.15mm, toprule=.15mm, opacitybacktitle=0.6, leftrule=.75mm]

Tenemos
\[ (y')^n + a_1(x, y)(y')^{n-1} + \dots + a_{n-1}(x, y)y' + a_n(x, y) = 0. \]

Resolviéndolo como un polinomio en \(y'\) de grado \(n\) igualado a cero
obtenemos
\[ (y' - f_1(x, y))(y' - f_2(x, y))\cdots(y' - f_n(x, y)) = 0. \]

Por tanto, las soluciones de la E. D. de partida serán las soluciones de
cada una de las ecuaciones \(y' - f_i(x, y) = 0\),
\(i = 1, 2, \dots, n\).

\end{tcolorbox}

\subsection*{Ejemplo 7}\label{ejemplo-7}
\addcontentsline{toc}{subsection}{Ejemplo 7}

\textbf{Resolver} \[ y^2((y')^2 + 1) = 1. \]

\textbf{Solución:}

Despejando \((y')^2\) queda \((y')^2 = \frac{1-y^2}{y^2}\), cuyas
soluciones algebraicas son \(y' = \pm \frac{\sqrt{1-y^2}}{y}\).

Vamos a resolver estas dos nuevas ecuaciones diferenciales a la vez
(ambas son de variables separadas). Si las ponemos como
\(\pm \frac{y \, dy}{\sqrt{1-y^2}} = dx\), integrando,
\(\mp \sqrt{1 - y^2} = x + C\).

Si elevamos al cuadrado ambos términos, podemos expresar conjuntamente
las dos familias de soluciones como \(1 - y^2 = (x + C)^2\), es decir:
\[(x + C)^2 + y^2 = 1\]

que son las \textbf{circunferencias con centro en el eje \(x\) y radio
1}.

\begin{tcolorbox}[enhanced jigsaw, breakable, colframe=quarto-callout-important-color-frame, left=2mm, coltitle=black, opacityback=0, colbacktitle=quarto-callout-important-color!10!white, bottomtitle=1mm, title=\textcolor{quarto-callout-important-color}{\faExclamation}\hspace{0.5em}{Soluciones singulares}, titlerule=0mm, arc=.35mm, bottomrule=.15mm, toptitle=1mm, colback=white, rightrule=.15mm, toprule=.15mm, opacitybacktitle=0.6, leftrule=.75mm]

Por último, es evidente que \(y = 1\) e \(y = -1\) también son
soluciones de la E. D., aunque no se encuentran entre las que acabamos
de hallar. Si atendemos a la interpretación geométrica de las ecuaciones
diferenciales, es lógico que estas dos rectas sean soluciones, ya que
son las \textbf{envolventes} de la familia de circunferencias que
satisfacen la E. D.

\end{tcolorbox}

\section{Obtención de la envolvente de una familia de
curvas}\label{obtenciuxf3n-de-la-envolvente-de-una-familia-de-curvas}

Recordemos qué es una envolvente: Dada una familia uniparamétrica de
curvas \(\{\varphi_\alpha\}_{\alpha \in A}\) en el plano, la envolvente
de la familia es una nueva curva \(\varphi\) tal que, en cada punto de
contacto de \(\varphi\) con alguna de las \(\varphi_\alpha\), la
tangente de \(\varphi\) y de \(\varphi_\alpha\) es la misma.

\subsection*{Familia de curvas
implícitas}\label{familia-de-curvas-impluxedcitas}
\addcontentsline{toc}{subsection}{Familia de curvas implícitas}

Supongamos que tenemos una familia de curvas expresadas implícitamente
como \(F(x, y, C) = 0\). Las envolventes se obtienen eliminando \(C\)
del sistema: \[
\begin{cases}
F(x, y, C) = 0 \\
F_C(x, y, C) = 0
\end{cases}
\] donde con \(F_C\) hemos denotado la derivada parcial de \(F\)
respecto a \(C\).

En el caso de las circunferencias del ejemplo anterior tendríamos: \[
\begin{cases}
(x + C)^2 + y^2 = 1 \\
2(x + C) = 0
\end{cases}
\]

Despejando en la segunda ecuación, \(x = -C\) y, sustituyendo en la
primera, \(y^2 = 1\). Es decir, las envolventes son las rectas
\(y = \pm 1\).

\subsection*{Familia de curvas
paramétricas}\label{familia-de-curvas-paramuxe9tricas}
\addcontentsline{toc}{subsection}{Familia de curvas paramétricas}

Supongamos que tenemos la familia de curvas en paramétricas: \[
\begin{cases}
x = x(t, C) \\
y = y(t, C)
\end{cases}
\]

Las envolventes se obtienen despejando \(C = C(t)\) en el jacobiano
igualado a cero: \[
\det \begin{pmatrix} x_C & y_C \\ x_t & y_t \end{pmatrix} = 0
\] y sustituyendo en las ecuaciones paramétricas.

\begin{tcolorbox}[enhanced jigsaw, breakable, colframe=quarto-callout-note-color-frame, left=2mm, coltitle=black, opacityback=0, colbacktitle=quarto-callout-note-color!10!white, bottomtitle=1mm, title=\textcolor{quarto-callout-note-color}{\faInfo}\hspace{0.5em}{Puntos singulares}, titlerule=0mm, arc=.35mm, bottomrule=.15mm, toptitle=1mm, colback=white, rightrule=.15mm, toprule=.15mm, opacitybacktitle=0.6, leftrule=.75mm]

Realmente, con estos procedimientos no sólo aparecen las envolventes,
sino también los lugares geométricos de \textbf{puntos singulares}, ya
sea de puntos de retroceso, de puntos de inflexión o de cúspides.

Los puntos de retroceso se llaman \textbf{puntos singulares esenciales}
y los otros \textbf{puntos singulares evitables}. En todo caso, después
de aplicar estos métodos, hay que comprobar siempre qué es lo que hemos
encontrado.

\end{tcolorbox}

\chapter{\texorpdfstring{Apartado 8: Ecuación de la forma
\(y = f(x, y')\)}{Apartado 8: Ecuación de la forma y = f(x, y\textquotesingle)}}\label{apartado-8-ecuaciuxf3n-de-la-forma-y-fx-y}

\section{Método general}\label{muxe9todo-general}

Como procedimiento general para intentar resolver este tipo de
ecuaciones, tomamos \(y' = p\) y derivamos \(y = f(x, y')\) respecto de
\(x\), con lo cual tenemos
\[ p = y' = f_x + f_{y'} \frac{dy'}{dx} = f_x(x, p) + f_{y'}(x, p)p'. \]

Cuando \(f\) tiene la forma adecuada, la nueva ecuación
\(p = f_x(x, p) + f_p(x, p)p'\) que hemos encontrado puede ser de alguno
de los tipos ya estudiados. Si éste es el caso, la resolvemos,
obteniendo su solución \(x = \phi(p, C)\).

Entonces, la solución de la E. D. de partida será: \[
\begin{cases}
x = \phi(p, C) \\
y = f(\phi(p, C), p)
\end{cases}
\] expresado como una familia de curvas en paramétricas.

\begin{tcolorbox}[enhanced jigsaw, breakable, colframe=quarto-callout-note-color-frame, left=2mm, coltitle=black, opacityback=0, colbacktitle=quarto-callout-note-color!10!white, bottomtitle=1mm, title=\textcolor{quarto-callout-note-color}{\faInfo}\hspace{0.5em}{Note}, titlerule=0mm, arc=.35mm, bottomrule=.15mm, toptitle=1mm, colback=white, rightrule=.15mm, toprule=.15mm, opacitybacktitle=0.6, leftrule=.75mm]

Puede resultar extraño pensar que el parámetro \(p\) vale precisamente
\(\frac{dy}{dx}\); pero esto no importa en absoluto, sino que puede
considerarse una simple curiosidad.

\end{tcolorbox}

\section{\texorpdfstring{8.1. Ecuación
\(y = f(y')\)}{8.1. Ecuación y = f(y\textquotesingle)}}\label{ecuaciuxf3n-y-fy}

En este caso, como \(f(x, y') = f(y')\), se tiene \(f_x(x, p) = 0\),
luego con el proceso descrito habremos obtenido la ecuación
\(p = f_p(p)p'\) o, lo que es lo mismo, \(p = f'(p)p'\).

Si \(p \neq 0\), esto lo podemos poner como
\(dx = \frac{f'(p)}{p} \, dp\), que es de variables separadas.
Integrándola, \(x = \int \frac{f'(p)}{p} \, dp = \phi(p) + C\), y las
soluciones de \(y = f(y')\) serán las curvas: \[
\begin{cases}
x = \phi(p) + C \\
y = f(p)
\end{cases}
\]

\begin{tcolorbox}[enhanced jigsaw, breakable, colframe=quarto-callout-important-color-frame, left=2mm, coltitle=black, opacityback=0, colbacktitle=quarto-callout-important-color!10!white, bottomtitle=1mm, title=\textcolor{quarto-callout-important-color}{\faExclamation}\hspace{0.5em}{Observación geométrica}, titlerule=0mm, arc=.35mm, bottomrule=.15mm, toptitle=1mm, colback=white, rightrule=.15mm, toprule=.15mm, opacitybacktitle=0.6, leftrule=.75mm]

Notar, a la vista de su representación paramétrica, que todas ellas
tienen la misma forma: se diferencian únicamente en un desplazamiento
horizontal.

\end{tcolorbox}

A partir de \(p = 0\) obtenemos la solución constante \(y = f(0)\).
Gráficamente, esto es una recta envolvente de las demás soluciones.

\section{\texorpdfstring{8.2. Ecuación de Lagrange:
\(y + x\varphi(y') + \psi(y') = 0\)}{8.2. Ecuación de Lagrange: y + x\textbackslash varphi(y\textquotesingle) + \textbackslash psi(y\textquotesingle) = 0}}\label{ecuaciuxf3n-de-lagrange-y-xvarphiy-psiy-0}

Si ponemos \(y' = p\) y derivamos respecto de \(x\) queda:
\[ p + \varphi(p) + x\varphi'(p)\frac{dp}{dx} + \psi'(p)\frac{dp}{dx} = 0 \]

que podemos escribir como:
\[(p + \varphi(p)) \, dx + x\varphi'(p) \, dp + \psi'(p) \, dp = 0\]

\subsection*{\texorpdfstring{Caso 1:
\(p + \varphi(p) \neq 0\)}{Caso 1: p + \textbackslash varphi(p) \textbackslash neq 0}}\label{caso-1-p-varphip-neq-0}
\addcontentsline{toc}{subsection}{Caso 1: \(p + \varphi(p) \neq 0\)}

Dividimos por \(p + \varphi(p)\), con lo que nos aparece la ecuación
lineal:
\[ \frac{dx}{dp} + \frac{\varphi'(p)}{p + \varphi(p)}x + \frac{\psi'(p)}{p + \varphi(p)} = 0 \]

en la que \(x\) actúa como función y \(p\) como variable. Si la
resolvemos, obtendremos como solución \(x = \phi(p, C)\). Con esto,
habremos encontrado para la E. D. de Lagrange las soluciones: \[
\begin{cases}
x = \phi(p, C) \\
y = -\phi(p, C)\varphi(p) - \psi(p)
\end{cases}
\] en forma de familia de curvas paramétricas.

\subsection*{Caso 2: Soluciones
singulares}\label{caso-2-soluciones-singulares-1}
\addcontentsline{toc}{subsection}{Caso 2: Soluciones singulares}

Supongamos que existe algún \(\lambda\) para el cual
\(\lambda + \varphi(\lambda) = 0\). Formalmente, tomamos
\(y' = \lambda\), sustituyendo en la E. D.,
\(y + x\varphi(\lambda) + \psi(\lambda) = 0\), es decir:
\[y = \lambda x - \psi(\lambda)\]

Esta recta es, efectivamente, solución de la ecuación de Lagrange. Estas
rectas decimos que son las \textbf{soluciones singulares} de la
ecuación.

\section{\texorpdfstring{8.3. Ecuación de Clairaut:
\(y - xy' + \psi(y') = 0\)}{8.3. Ecuación de Clairaut: y - xy\textquotesingle{} + \textbackslash psi(y\textquotesingle) = 0}}\label{ecuaciuxf3n-de-clairaut-y---xy-psiy-0}

Éste es un caso particular de ecuación de Lagrange con
\(\varphi(y') = -y'\). Aquí, como \(\varphi(\lambda) + \lambda = 0\)
para todo \(\lambda \in \mathbb{R}\), no existen las soluciones que, en
la ecuación de Lagrange, encontrábamos por el método general; sólo
aparecen rectas.

\begin{tcolorbox}[enhanced jigsaw, breakable, colframe=quarto-callout-tip-color-frame, left=2mm, coltitle=black, opacityback=0, colbacktitle=quarto-callout-tip-color!10!white, bottomtitle=1mm, title=\textcolor{quarto-callout-tip-color}{\faLightbulb}\hspace{0.5em}{Solución general de Clairaut}, titlerule=0mm, arc=.35mm, bottomrule=.15mm, toptitle=1mm, colback=white, rightrule=.15mm, toprule=.15mm, opacitybacktitle=0.6, leftrule=.75mm]

Tenemos como soluciones de la E. D. toda la familia de rectas:
\[y = \lambda x - \psi(\lambda), \quad \lambda \in \mathbb{R}\]

\end{tcolorbox}

\subsection*{Solución singular
(envolvente)}\label{soluciuxf3n-singular-envolvente}
\addcontentsline{toc}{subsection}{Solución singular (envolvente)}

Existe una solución singular, la envolvente de este haz de rectas. Para
ello, planteamos el sistema: \[
\begin{cases}
y = \lambda x - \psi(\lambda) \\
0 = x - \psi'(\lambda)
\end{cases}
\]

donde la segunda ecuación es la primera derivada respecto al parámetro
\(\lambda\). La envolvente queda expresada en forma paramétrica como: \[
\begin{cases}
x = \psi'(\lambda) \\
y = \lambda \psi'(\lambda) - \psi(\lambda)
\end{cases}
\]

\begin{tcolorbox}[enhanced jigsaw, breakable, colframe=quarto-callout-note-color-frame, left=2mm, coltitle=black, opacityback=0, colbacktitle=quarto-callout-note-color!10!white, bottomtitle=1mm, title=\textcolor{quarto-callout-note-color}{\faInfo}\hspace{0.5em}{Verificación}, titlerule=0mm, arc=.35mm, bottomrule=.15mm, toptitle=1mm, colback=white, rightrule=.15mm, toprule=.15mm, opacitybacktitle=0.6, leftrule=.75mm]

Podemos garantizar que esta curva es una envolvente y no un lugar
geométrico de puntos singulares. En cada punto
\((x(\lambda), y(\lambda))\) de la curva, esta interseca precisamente a
la recta \(y = \lambda x - \psi(\lambda)\) del haz; y la pendiente de
esa recta en el punto de intersección coincide con la de la curva:
\[ \frac{dy}{dx} = \frac{dy/d\lambda}{dx/d\lambda} = \frac{\psi'(\lambda) + \lambda\psi''(\lambda) - \psi'(\lambda)}{\psi''(\lambda)} = \lambda \]

\end{tcolorbox}

\begin{tcolorbox}[enhanced jigsaw, breakable, colframe=quarto-callout-tip-color-frame, left=2mm, coltitle=black, opacityback=0, colbacktitle=quarto-callout-tip-color!10!white, bottomtitle=1mm, title=\textcolor{quarto-callout-tip-color}{\faLightbulb}\hspace{0.5em}{RECETA 8. Ecuación de la forma \(y = f(x, y')\)}, titlerule=0mm, arc=.35mm, bottomrule=.15mm, toptitle=1mm, colback=white, rightrule=.15mm, toprule=.15mm, opacitybacktitle=0.6, leftrule=.75mm]

En general, se toma \(y' = p\) y se deriva la ecuación \(y = f(x, y')\)
respecto de \(x\). Si \(f\) tiene la forma adecuada, a la nueva E. D. se
le puede aplicar alguno de los métodos ya estudiados.

\textbf{8.1.} Cuando \(y = f(y')\), la ecuación que se obtiene es de
variables separadas.

\textbf{8.2.} Ecuación de Lagrange: \(y + x\varphi(y') + \psi(y') = 0\).
Se reduce a una ecuación lineal con \(x\) como función y \(p\) como
variable. Además, para los \(\lambda\) tales que
\(\lambda + \varphi(\lambda) = 0\) se obtienen como soluciones las
rectas \(y = \lambda x - \psi(\lambda)\).

\textbf{8.3.} Ecuación de Clairaut: \(y - xy' + \psi(y') = 0\). Es un
caso particular de ecuación de Lagrange en el que sólo aparecen rectas
(y su envolvente).

\end{tcolorbox}

\subsection*{Ejemplo 8.3: Ecuación de
Clairaut}\label{ejemplo-8.3-ecuaciuxf3n-de-clairaut}
\addcontentsline{toc}{subsection}{Ejemplo 8.3: Ecuación de Clairaut}

\textbf{Resolver} \[ y = y'x - 2(y')^2. \]

\textbf{Solución:}

Estamos ante una ecuación de Clairaut \(y - xy' + \psi(y') = 0\) con
\(\psi(y') = 2(y')^2\).

Sin más que tomar \(y' = \lambda\), las rectas
\(y = \lambda x - 2\lambda^2\) son solución de la E. D. para cada
\(\lambda \in \mathbb{R}\).

\textbf{Calculemos su envolvente:} en el sistema \[
\begin{cases}
y = \lambda x - 2\lambda^2 \\
0 = x - 4\lambda
\end{cases}
\]

Despejamos \(\lambda = \frac{x}{4}\) en la segunda ecuación y,
sustituyendo en la primera, encontramos que la envolvente es la
parábola: \[y = \frac{x^2}{8}\]

\chapter{\texorpdfstring{Apartado 9: Ecuación de la forma
\(x = f(y, y')\)}{Apartado 9: Ecuación de la forma x = f(y, y\textquotesingle)}}\label{apartado-9-ecuaciuxf3n-de-la-forma-x-fy-y}

\section{Método general}\label{muxe9todo-general-1}

Estas ecuaciones son similares a las que aparecen en el apartado 8 pero
con el papel de \(x\) e \(y\) intercambiado. En general, para intentar
resolverlas, tomamos \(y' = p\) y derivamos \(x = f(y, y')\) (o, si se
prefiere interpretarlo así, \(x = f(y, p)\)) respecto de \(y\) (en lugar
de respecto a \(x\), como hacíamos en el apartado 8).

\begin{tcolorbox}[enhanced jigsaw, breakable, colframe=quarto-callout-warning-color-frame, left=2mm, coltitle=black, opacityback=0, colbacktitle=quarto-callout-warning-color!10!white, bottomtitle=1mm, title=\textcolor{quarto-callout-warning-color}{\faExclamationTriangle}\hspace{0.5em}{¡Atención!}, titlerule=0mm, arc=.35mm, bottomrule=.15mm, toptitle=1mm, colback=white, rightrule=.15mm, toprule=.15mm, opacitybacktitle=0.6, leftrule=.75mm]

Esto requiere tener un poco más de cuidado, además de usar
\(\frac{dx}{dy} = \frac{1}{dy/dx} = \frac{1}{p}\).

\end{tcolorbox}

Así:
\[ \frac{1}{p} = \frac{dx}{dy} = f_y(y, p) + f_p(y, p)\frac{dp}{dy} \]

Según como sea \(f\), esta nueva ecuación
\(\frac{1}{p} = f_y(y, p) + f_p(y, p)\frac{dp}{dy}\) a la que hemos
reducido la que teníamos responde a alguno de los tipos previamente
estudiados, luego podríamos resolverla.

Si logramos hacerlo, obtenemos su solución \(y = \phi(p, C)\). Con esto,
la solución de la ecuación de partida será la familia de curvas
paramétricas: \[
\begin{cases}
x = f(\phi(p, C), p) \\
y = \phi(p, C)
\end{cases}
\]

\section{Casos especiales}\label{casos-especiales-1}

Pero no podemos asegurar que, para una función \(f\) cualquiera, sepamos
resolver la ecuación intermedia que nos aparece. Se pueden estudiar
casos similares a los de la forma \(y = f(x, y')\) en los que se
garantiza que el método no quedará interrumpido.

Los tres casos que merece la pena distinguir son los siguientes:

\subsection*{\texorpdfstring{9.1. Ecuación
\(x = f(y')\)}{9.1. Ecuación x = f(y\textquotesingle)}}\label{ecuaciuxf3n-x-fy}
\addcontentsline{toc}{subsection}{9.1. Ecuación \(x = f(y')\)}

Similar al caso 8.1, pero derivando respecto a \(y\).

\subsection*{\texorpdfstring{9.2. Ecuación
\(x + y\varphi(y') + \psi(y') = 0\)}{9.2. Ecuación x + y\textbackslash varphi(y\textquotesingle) + \textbackslash psi(y\textquotesingle) = 0}}\label{ecuaciuxf3n-x-yvarphiy-psiy-0}
\addcontentsline{toc}{subsection}{9.2. Ecuación
\(x + y\varphi(y') + \psi(y') = 0\)}

Similar al caso 8.2 (ecuación de Lagrange), pero con \(x\) e \(y\)
intercambiados.

\subsection*{\texorpdfstring{9.3. Ecuación
\(x - \frac{y}{y'} + \psi(y') = 0\)}{9.3. Ecuación x - \textbackslash frac\{y\}\{y\textquotesingle\} + \textbackslash psi(y\textquotesingle) = 0}}\label{ecuaciuxf3n-x---fracyy-psiy-0}
\addcontentsline{toc}{subsection}{9.3. Ecuación
\(x - \frac{y}{y'} + \psi(y') = 0\)}

Similar al caso 8.3 (ecuación de Clairaut), pero con \(x\) e \(y\)
intercambiados.

\begin{tcolorbox}[enhanced jigsaw, breakable, colframe=quarto-callout-note-color-frame, left=2mm, coltitle=black, opacityback=0, colbacktitle=quarto-callout-note-color!10!white, bottomtitle=1mm, title=\textcolor{quarto-callout-note-color}{\faInfo}\hspace{0.5em}{Ejercicio}, titlerule=0mm, arc=.35mm, bottomrule=.15mm, toptitle=1mm, colback=white, rightrule=.15mm, toprule=.15mm, opacitybacktitle=0.6, leftrule=.75mm]

Se deja al lector que, como ejercicio, describa el método de resolución
de estos tres tipos de ecuaciones. Simplemente hay que dedicarse a
modificar, con cuidado, el proceso utilizado en los correspondientes
tipos 8.1, 8.2 y 8.3, recordando que ahora hay que derivar respecto de
\(y\) en lugar de respecto de \(x\).

\end{tcolorbox}

\begin{tcolorbox}[enhanced jigsaw, breakable, colframe=quarto-callout-tip-color-frame, left=2mm, coltitle=black, opacityback=0, colbacktitle=quarto-callout-tip-color!10!white, bottomtitle=1mm, title=\textcolor{quarto-callout-tip-color}{\faLightbulb}\hspace{0.5em}{RECETA 9. Ecuación de la forma \(x = f(y, y')\)}, titlerule=0mm, arc=.35mm, bottomrule=.15mm, toptitle=1mm, colback=white, rightrule=.15mm, toprule=.15mm, opacitybacktitle=0.6, leftrule=.75mm]

En general, se toma \(y' = p\) y se deriva la ecuación \(x = f(y, y')\)
respecto de \(y\). Según como sea \(f\), la nueva E. D. que así se
obtiene es ya conocida, procediéndose a su resolución. Se pueden
estudiar casos similares a los de la forma \(y = f(x, y')\).

\end{tcolorbox}

\subsection*{Ejemplo 9}\label{ejemplo-9}
\addcontentsline{toc}{subsection}{Ejemplo 9}

\textbf{Resolver} \[ x = (y')^3 + y'. \]

\textbf{Solución:}

Si tomamos \(\frac{dy}{dx} = y' = p\) y derivamos \(x = p^3 + p\)
respecto de \(y\) tenemos:
\[ \frac{1}{p} = \frac{1}{dy/dx} = \frac{dx}{dy} = 3p^2 \frac{dp}{dy} + \frac{dp}{dy} \]

que podemos poner como \(dy = (3p^3 + p) \, dp\), ecuación en variables
separadas.

Integrándola, \(y = \frac{3}{4}p^4 + \frac{1}{2}p^2 + C\).

\begin{tcolorbox}[enhanced jigsaw, breakable, colframe=quarto-callout-important-color-frame, left=2mm, coltitle=black, opacityback=0, colbacktitle=quarto-callout-important-color!10!white, bottomtitle=1mm, title=\textcolor{quarto-callout-important-color}{\faExclamation}\hspace{0.5em}{Solución}, titlerule=0mm, arc=.35mm, bottomrule=.15mm, toptitle=1mm, colback=white, rightrule=.15mm, toprule=.15mm, opacitybacktitle=0.6, leftrule=.75mm]

Por lo tanto, las soluciones de la ecuación \(x = (y')^3 + y'\) son la
familia de curvas paramétricas: \[
\begin{cases}
x = p^3 + p \\
y = \frac{3}{4}p^4 + \frac{1}{2}p^2 + C
\end{cases}
\]

\end{tcolorbox}

\chapter{\texorpdfstring{Apartado 10: Ecuación de la forma
\(F(y, y') = 0\)}{Apartado 10: Ecuación de la forma F(y, y\textquotesingle) = 0}}\label{apartado-10-ecuaciuxf3n-de-la-forma-fy-y-0}

\section{Método general}\label{muxe9todo-general-2}

Para intentar encontrar sus soluciones, consideremos la curva
\(F(\alpha, \beta) = 0\). El método de resolución requiere que hayamos
logrado encontrar previamente una \textbf{representación paramétrica} de
la curva, esto es, \(\alpha = \varphi(t)\) y \(\beta = \psi(t)\) tal que
\(F(\varphi(t), \psi(t)) = 0\).

Si así ha sido, vamos ahora a efectuar el cambio de función \(y\) por
\(t\) mediante: \[y = \varphi(t), \quad y' = \psi(t)\]

Entonces, derivando \(y = \varphi(t)\) respecto de \(x\) tenemos
\(y' = \varphi'(t)\frac{dt}{dx}\) que, al ser \(y' = \psi(t)\), puede
escribirse como: \[ \psi(t) = \varphi'(t)\frac{dt}{dx} \]

Esta es una ecuación en \textbf{variables separadas}.

\section{Resolución según casos}\label{resoluciuxf3n-seguxfan-casos}

\subsection*{\texorpdfstring{Caso 1:
\(\psi(t) \neq 0\)}{Caso 1: \textbackslash psi(t) \textbackslash neq 0}}\label{caso-1-psit-neq-0}
\addcontentsline{toc}{subsection}{Caso 1: \(\psi(t) \neq 0\)}

Podemos poner \(dx = \frac{\varphi'(t)}{\psi(t)} \, dt\) e, integrando:
\[x = \int \frac{\varphi'(t)}{\psi(t)} \, dt + C\]

Como consecuencia, la familia de curvas paramétricas: \[
\begin{cases}
x = \int \frac{\varphi'(t)}{\psi(t)} \, dt + C \\
y = \varphi(t)
\end{cases}
\] son soluciones de la E. D. de partida.

\subsection*{\texorpdfstring{Caso 2: Soluciones singulares
(\(\psi(t_0) = 0\))}{Caso 2: Soluciones singulares (\textbackslash psi(t\_0) = 0)}}\label{caso-2-soluciones-singulares-psit_0-0}
\addcontentsline{toc}{subsection}{Caso 2: Soluciones singulares
(\(\psi(t_0) = 0\))}

Si existe algún \(t_0\) para el que \(\psi(t_0) = 0\), formalmente
efectuamos el siguiente razonamiento: \[0 = \varphi'(t)\frac{dt}{dx}\]

así que \(\varphi'(t) = 0\) y consecuentemente
\(\varphi(t) = \text{cte.} = \varphi(t_0)\), lo que nos conduciría a la
solución \(y = \varphi(t_0)\).

Y, en efecto, podemos comprobar con rigor que esta \textbf{recta
horizontal} es solución de la E. D. sin más que sustituir en ella:
\[F(y, y') = F(\varphi(t_0), 0) = F(\varphi(t_0), \psi(t_0)) = 0\]

\begin{tcolorbox}[enhanced jigsaw, breakable, colframe=quarto-callout-tip-color-frame, left=2mm, coltitle=black, opacityback=0, colbacktitle=quarto-callout-tip-color!10!white, bottomtitle=1mm, title=\textcolor{quarto-callout-tip-color}{\faLightbulb}\hspace{0.5em}{RECETA 10. Ecuación de la forma \(F(y, y') = 0\)}, titlerule=0mm, arc=.35mm, bottomrule=.15mm, toptitle=1mm, colback=white, rightrule=.15mm, toprule=.15mm, opacitybacktitle=0.6, leftrule=.75mm]

Consideramos la curva \(F(\alpha, \beta) = 0\). Si encontramos una
representación paramétrica \(\alpha = \varphi(t)\), \(\beta = \psi(t)\),
\(F(\varphi(t), \psi(t)) = 0\), se hace el cambio de función \(y\) por
\(t\) mediante \(y = \varphi(t)\), \(y' = \psi(t)\). Así, derivando
\(y = \varphi(t)\) respecto de \(x\), aparece una ecuación en variables
separadas.

\end{tcolorbox}

\subsection*{Ejemplo 10}\label{ejemplo-10}
\addcontentsline{toc}{subsection}{Ejemplo 10}

\textbf{Resolver} \[ y^{2/3} + (y')^{2/3} = 1. \]

\textbf{Solución:}

Si tomamos \(y = \cos^3 t\), \(y' = \operatorname{sen}^3 t\), es claro
que: \[(\cos^3 t)^{2/3} + (\operatorname{sen}^3 t)^{2/3} = 1\]

Derivando \(y = \cos^3 t\) tenemos
\(y' = -3 \cos^2 t \operatorname{sen} t \frac{dt}{dx}\), lo que, usando
\(y' = \operatorname{sen}^3 t\), resulta ser:
\[\operatorname{sen}^3 t = -3 \cos^2 t \operatorname{sen} t \frac{dt}{dx}\]

\textbf{Si suponemos} \(\operatorname{sen} t \neq 0\), tenemos:
\[dx = \frac{-3 \cos^2 t}{\operatorname{sen}^2 t} \, dt = -3 \operatorname{cotg}^2 t \, dt\]

Integrando: \begin{align*}
x &= -3 \int \operatorname{cotg}^2 t \, dt \\
&= -3 \int (1 + \operatorname{cotg}^2 t) \, dt + 3 \int dt \\
&= 3 \operatorname{cotg} t + 3t + C
\end{align*}

Por tanto, las curvas paramétricas: \[
\begin{cases}
x = 3 \operatorname{cotg} t + 3t + C \\
y = \cos^3 t
\end{cases}
\] son soluciones de la E. D. de partida.

\begin{tcolorbox}[enhanced jigsaw, breakable, colframe=quarto-callout-important-color-frame, left=2mm, coltitle=black, opacityback=0, colbacktitle=quarto-callout-important-color!10!white, bottomtitle=1mm, title=\textcolor{quarto-callout-important-color}{\faExclamation}\hspace{0.5em}{Soluciones singulares}, titlerule=0mm, arc=.35mm, bottomrule=.15mm, toptitle=1mm, colback=white, rightrule=.15mm, toprule=.15mm, opacitybacktitle=0.6, leftrule=.75mm]

Por último, para los \(t\) tales que \(\operatorname{sen}^3 t = 0\), lo
cual ocurre cuando \(t = k\pi\), \(k \in \mathbb{Z}\), aparecen como
soluciones singulares las rectas horizontales \(y = \cos^3(k\pi)\), es
decir: \[y = 1 \quad \text{e} \quad y = -1\]

\end{tcolorbox}

\part{Parte III: Reducción de Orden}

\chapter{\texorpdfstring{Apartado 11:
\(F(x, y^{(k)}, \dots, y^{(n)}) = 0\)}{Apartado 11: F(x, y\^{}\{(k)\}, \textbackslash dots, y\^{}\{(n)\}) = 0}}\label{apartado-11-fx-yk-dots-yn-0}

\section{\texorpdfstring{Reducción de orden: ausencia de
\(y\)}{Reducción de orden: ausencia de y}}\label{reducciuxf3n-de-orden-ausencia-de-y}

Tenemos una ecuación en la que no aparece la variable dependiente \(y\)
(y además, si \(k > 1\), tampoco sus derivadas hasta el orden
\(k - 1\)).

Es evidente que el cambio \(y^{(k)} = z\) la convierte en:
\[ F(x, z, \dots, z^{(n-k)}) = 0 \]

que es una E. D. de orden \(n - k\).

\section{Procedimiento completo}\label{procedimiento-completo}

Si logramos resolverla, obtenemos que su solución será
\(z = \phi(x, C_1, \dots, C_{n-k})\), una familia dependiente de
\(n - k\) constantes.

Entonces, al ser \(y^{(k)} = z\), para tener las soluciones de la
ecuación original bastará con integrar \(k\) veces
\(\phi(x, C_1, \dots, C_{n-k})\) respecto de \(x\), es decir:
\[ y = \underbrace{\int \cdots \int}_{k \text{ veces}} \phi(x, C_1, \dots, C_{n-k}) \underbrace{dx \, dx \dots dx}_{k \text{ veces}} \]

lo cual introducirá \(k\) nuevas constantes en la solución general. De
este modo, dicha solución general tendrá la forma:
\[ y = \underbrace{\int \cdots \int}_{k \text{ veces}} \phi(x, C_1, \dots, C_{n-k}) \underbrace{dx \, dx \dots dx}_{k \text{ veces}} + C_{n-k+1}x^{k-1} + \cdots + C_{n-1}x + C_n \]

\begin{tcolorbox}[enhanced jigsaw, breakable, colframe=quarto-callout-tip-color-frame, left=2mm, coltitle=black, opacityback=0, colbacktitle=quarto-callout-tip-color!10!white, bottomtitle=1mm, title=\textcolor{quarto-callout-tip-color}{\faLightbulb}\hspace{0.5em}{RECETA 11. Ecuación de la forma \(F(x, y^{(k)}, \dots, y^{(n)}) = 0\)}, titlerule=0mm, arc=.35mm, bottomrule=.15mm, toptitle=1mm, colback=white, rightrule=.15mm, toprule=.15mm, opacitybacktitle=0.6, leftrule=.75mm]

Mediante el cambio \(y^{(k)} = z\) se convierte en una ecuación de orden
\(n - k\).

\end{tcolorbox}

\subsection*{Ejemplo 11}\label{ejemplo-11}
\addcontentsline{toc}{subsection}{Ejemplo 11}

\textbf{Resolver} \[ y'' - xy''' + (y''')^3 = 0. \]

\textbf{Solución:}

Tomemos \(y'' = z\), con lo que la ecuación se transforma en
\(z - xz' + (z')^3 = 0\), que es de \textbf{Clairaut}.

Sin más que sustituir \(z' = \lambda\), las soluciones de ésta son las
rectas: \[z = \lambda x - \lambda^3, \quad \lambda \in \mathbb{R}\]

\textbf{Calculemos las envolventes:} en el sistema \[
\begin{cases}
z = \lambda x - \lambda^3 \\
0 = x - 3\lambda^2
\end{cases}
\]

Despejamos \(\lambda = \pm\sqrt{\frac{x}{3}}\) en la segunda ecuación,
con lo que, sustituyendo en la primera, encontramos las dos envolventes:
\[z = \pm\frac{2x}{3\sqrt{3}}\sqrt{\frac{x}{3}}\]

\textbf{Una vez resuelta completamente la ecuación de Clairaut, hallemos
las soluciones de la original:}

Como \(y'' = z\), a partir de \(z = \lambda x - \lambda^3\), integrando
dos veces:
\[ y = \int \left( \int (\lambda x - \lambda^3) \, dx \right) dx = \frac{\lambda x^3}{6} - \frac{\lambda^3 x^2}{2} + C_1 x + C_2 \]

Cambiando la notación de las constantes:
\[y = K_1x^3 - 108K_1^3 x^2 + K_2 x + K_3\]

Por último, de \(z = \pm\frac{2}{3\sqrt{3}}x^{3/2}\) obtenemos:
\[ y = \int \left( \int \frac{\pm 2}{3\sqrt{3}}x^{3/2} \, dx \right) dx = \frac{\pm 8x^{7/2}}{105\sqrt{3}} + C_1 x + C_2 \]

\begin{tcolorbox}[enhanced jigsaw, breakable, colframe=quarto-callout-important-color-frame, left=2mm, coltitle=black, opacityback=0, colbacktitle=quarto-callout-important-color!10!white, bottomtitle=1mm, title=\textcolor{quarto-callout-important-color}{\faExclamation}\hspace{0.5em}{Soluciones finales}, titlerule=0mm, arc=.35mm, bottomrule=.15mm, toptitle=1mm, colback=white, rightrule=.15mm, toprule=.15mm, opacitybacktitle=0.6, leftrule=.75mm]

Es decir, las dos familias de curvas:
\[y = \frac{8x^{7/2}}{105\sqrt{3}} + C_1 x + C_2\]
\[y = -\frac{8x^{7/2}}{105\sqrt{3}} + C_1 x + C_2\]

\end{tcolorbox}

\chapter{Apartado 11': Ecuaciones Lineales de Orden
Superior}\label{apartado-11-ecuaciones-lineales-de-orden-superior}

\section{\texorpdfstring{Ecuaciones Lineales de Orden
\(n > 1\)}{Ecuaciones Lineales de Orden n \textgreater{} 1}}\label{ecuaciones-lineales-de-orden-n-1}

Supongamos que tenemos una ecuación:
\[ a_n(x)y^{(n)} + a_{n-1}(x)y^{(n-1)} + \cdots + a_1(x)y' + a_0(x)y = g(x) \]

con \(n > 1\).

\begin{tcolorbox}[enhanced jigsaw, breakable, colframe=quarto-callout-note-color-frame, left=2mm, coltitle=black, opacityback=0, colbacktitle=quarto-callout-note-color!10!white, bottomtitle=1mm, title=\textcolor{quarto-callout-note-color}{\faInfo}\hspace{0.5em}{Observación importante}, titlerule=0mm, arc=.35mm, bottomrule=.15mm, toptitle=1mm, colback=white, rightrule=.15mm, toprule=.15mm, opacitybacktitle=0.6, leftrule=.75mm]

Existe una teoría general bien establecida sobre ecuaciones
diferenciales lineales de cualquier orden. Pero éste es un estudio
esencialmente teórico: \textbf{no es posible describir la solución
general de una E. D. lineal de orden 2 o superior por medio de
cuadraturas}.

Sin embargo, sí que se calculan las soluciones de forma muy
satisfactoria cuando nos encontramos ante ecuaciones lineales con
coeficientes constantes.

\end{tcolorbox}

\section{Método de reducción de
orden}\label{muxe9todo-de-reducciuxf3n-de-orden}

Este método radica en haber encontrado previamente una \textbf{solución
particular} \(y_n(x)\) de la lineal homogénea asociada:
\[a_n(x)y^{(n)} + a_{n-1}(x)y^{(n-1)} + \cdots + a_1(x)y' + a_0(x)y = 0\]

Si hemos logrado encontrar tal solución particular, efectuamos el cambio
de función \(y\) por \(z\) dado por: \[y(x) = y_n(x)z(x)\]

\subsection*{Procedimiento}\label{procedimiento}
\addcontentsline{toc}{subsection}{Procedimiento}

Derivando sucesivamente: \begin{align*}
y'(x) &= y'_n(x)z(x) + y_n(x)z'(x) \\
y''(x) &= y''_n(x)z(x) + 2y'_n(x)z'(x) + y_n(x)z''(x) \\
&\vdots \\
y^{(n)}(x) &= y_n^{(n)}(x)z(x) + \cdots + \binom{n}{k}y_n^{(k)}(x)z^{(n-k)}(x) + \cdots + y_n(x)z^{(n)}(x)
\end{align*}

Sustituyendo en la lineal y sacando factor común los \(z^{(k)}\), el
coeficiente de \(z\) es 0 por ser precisamente \(y_n\) solución de la
ecuación lineal homogénea. Así, podemos poner:
\[ b_n(x)z^{(n)} + b_{n-1}(x)z^{(n-1)} + \cdots + b_1(x)z' = g(x) \]

Si en ella hacemos \(z' = u\) aparece:
\[ b_n(x)u^{(n-1)} + b_{n-1}(x)u^{(n-2)} + \cdots + b_1(x)u = g(x) \]

que es una \textbf{ecuación lineal de orden \(n-1\)}.

\begin{tcolorbox}[enhanced jigsaw, breakable, colframe=quarto-callout-tip-color-frame, left=2mm, coltitle=black, opacityback=0, colbacktitle=quarto-callout-tip-color!10!white, bottomtitle=1mm, title=\textcolor{quarto-callout-tip-color}{\faLightbulb}\hspace{0.5em}{RECETA 11'. Ecuaciones lineales de orden superior}, titlerule=0mm, arc=.35mm, bottomrule=.15mm, toptitle=1mm, colback=white, rightrule=.15mm, toprule=.15mm, opacitybacktitle=0.6, leftrule=.75mm]

Son de la forma:
\[ a_n(x)y^{(n)} + a_{n-1}(x)y^{(n-1)} + \cdots + a_1(x)y' + a_0(x)y = g(x) \]

Si logramos encontrar alguna solución \(y_n(x)\) de la lineal homogénea
asociada, el cambio de función \(y = y_n z\) hace que la lineal se
transforme en una del tipo anterior, cuyo orden se puede reducir. Así,
aparece una nueva ecuación lineal, esta vez de orden \(n-1\).

\end{tcolorbox}

\section{Resultados generales}\label{resultados-generales}

\subsection*{Estructura de la solución
general}\label{estructura-de-la-soluciuxf3n-general}
\addcontentsline{toc}{subsection}{Estructura de la solución general}

La solución general de la lineal de orden \(n\) puede expresarse como:
\[ y(x) = y_p(x) + C_1 y_1(x) + \cdots + C_{n-1}y_{n-1}(x) + C_n y_n(x) \]

donde \(y_p(x)\) es una solución particular de la lineal.

\begin{tcolorbox}[enhanced jigsaw, breakable, colframe=quarto-callout-important-color-frame, left=2mm, coltitle=black, opacityback=0, colbacktitle=quarto-callout-important-color!10!white, bottomtitle=1mm, title=\textcolor{quarto-callout-important-color}{\faExclamation}\hspace{0.5em}{Important}, titlerule=0mm, arc=.35mm, bottomrule=.15mm, toptitle=1mm, colback=white, rightrule=.15mm, toprule=.15mm, opacitybacktitle=0.6, leftrule=.75mm]

La solución general de la lineal puede expresarse como una solución
particular de la lineal más la solución general de la lineal homogénea
asociada.

\end{tcolorbox}

\subsection*{Método de variación de las
constantes}\label{muxe9todo-de-variaciuxf3n-de-las-constantes}
\addcontentsline{toc}{subsection}{Método de variación de las constantes}

También existe el método de variación de las constantes para E. D.
lineales de orden \(n\), destinado a resolver la ecuación lineal cuando
se conoce la solución general de la homogénea asociada,
\(y(x) = C_1 y_1(x) + C_2 y_2(x) + \cdots + C_n y_n(x)\).

En estas condiciones, se busca la solución de la lineal de la forma:
\[ y(x) = C_1(x)y_1(x) + C_2(x)y_2(x) + \cdots + C_n(x)y_n(x) \]

para ciertas funciones \(C_k(x)\) a determinar.

\chapter{\texorpdfstring{Apartado 12:
\(F(y, y', \ldots, y^{(n)}) = 0\)}{Apartado 12: F(y, y\textquotesingle, \textbackslash ldots, y\^{}\{(n)\}) = 0}}\label{apartado-12-fy-y-ldots-yn-0}

\section{\texorpdfstring{Reducción de orden: ausencia de
\(x\)}{Reducción de orden: ausencia de x}}\label{reducciuxf3n-de-orden-ausencia-de-x}

En esta ecuación no aparece explícitamente la variable independiente
\(x\). Vamos a ver cómo se reduce de orden si efectuamos el cambio
\(y' = p\) y la transformamos en una nueva en la que aparecerán \(y\),
\(p\) y las derivadas de \(p\) respecto de \(y\).

\begin{tcolorbox}[enhanced jigsaw, breakable, colframe=quarto-callout-warning-color-frame, left=2mm, coltitle=black, opacityback=0, colbacktitle=quarto-callout-warning-color!10!white, bottomtitle=1mm, title=\textcolor{quarto-callout-warning-color}{\faExclamationTriangle}\hspace{0.5em}{¡Atención!}, titlerule=0mm, arc=.35mm, bottomrule=.15mm, toptitle=1mm, colback=white, rightrule=.15mm, toprule=.15mm, opacitybacktitle=0.6, leftrule=.75mm]

Las derivadas de \(p\) son \textbf{respecto de \(y\)}, no respecto de
\(x\).

\end{tcolorbox}

\section{Derivadas sucesivas}\label{derivadas-sucesivas}

Si denotamos \(p' = \frac{dp}{dy}\), \(p'' = \frac{d^2p}{dy^2}\),
\ldots, y hacemos uso de la regla de la cadena, entonces:

\[y' = p\]

\[ y'' = \frac{dy'}{dx} = \frac{dp}{dx} = \frac{dp}{dy}\frac{dy}{dx} = p'p \]

\[ y''' = \frac{dy''}{dx} = \frac{d}{dx}(p'p) = \frac{d}{dy}(p'p)\frac{dy}{dx} = [p''p + (p')^2]p \]

y así sucesivamente. Nótese que en la expresión de cada \(y^{(k)}\) sólo
aparecen derivadas de \(p\) hasta el orden \(k-1\).

\section{Transformación de la
ecuación}\label{transformaciuxf3n-de-la-ecuaciuxf3n}

Con esto, sustituyendo en \(F\) los valores que hemos encontrado para
\(y'\), \(y''\), \ldots, \(y^{(n)}\), la E. D. original se transforma en
una de la forma:
\[ G\left( y, \frac{dp}{dy}, \ldots, \frac{d^{n-1}p}{dy^{n-1}} \right) = 0 \]

cuyo orden es \(n-1\).

Si logramos resolver esta nueva ecuación encontraremos que, en general,
su solución será de la forma \(p = \phi(y, C_1, \ldots, C_{n-1})\),
dependiente de \(n-1\) constantes.

Ahora, devolviendo a \(p\) su valor original, es decir,
\(p = \frac{dy}{dx}\), obtenemos:
\[\frac{dy}{dx} = \phi(y, C_1, \ldots, C_{n-1})\]

que es otra ecuación, esta vez de primer orden, que también hay que
resolver (lo cual, además, añade una nueva constante). Las soluciones de
esta última serán, obviamente, las mismas que las de la E. D. de
partida.

\begin{tcolorbox}[enhanced jigsaw, breakable, colframe=quarto-callout-tip-color-frame, left=2mm, coltitle=black, opacityback=0, colbacktitle=quarto-callout-tip-color!10!white, bottomtitle=1mm, title=\textcolor{quarto-callout-tip-color}{\faLightbulb}\hspace{0.5em}{RECETA 12. Ecuación de la forma \(F(y, y', \ldots, y^{(n)}) = 0\)}, titlerule=0mm, arc=.35mm, bottomrule=.15mm, toptitle=1mm, colback=white, rightrule=.15mm, toprule=.15mm, opacitybacktitle=0.6, leftrule=.75mm]

Hacemos el cambio \(y' = p\) y transformamos la E. D. en una nueva
dependiendo de \(y\), \(p\) y las derivadas de \(p\) respecto de \(y\).
Ésta es de orden \(n-1\).

\end{tcolorbox}

\subsection*{Ejemplo 12}\label{ejemplo-12}
\addcontentsline{toc}{subsection}{Ejemplo 12}

\textbf{Resolver} \[ 2yy'' = (y')^2 + 1. \]

\textbf{Solución:}

Tomando \(y' = p\), y tal como hemos comprobado al explicar el método,
se tiene \(y'' = p'p\) donde \(p' = \frac{dp}{dy}\).

Entonces, al sustituir en la E. D., ésta se nos transforma en
\(2yp'p = p^2 + 1\), es decir:
\[\frac{2p \, dp}{p^2 + 1} = \frac{dy}{y}\]

que es de variables separadas.

Si integramos: \[\log(p^2 + 1) = \log(C_1 y)\]

de donde \(p^2 + 1 = C_1 y\), y por tanto \(p = \pm\sqrt{C_1 y - 1}\).

Volvemos ahora a restaurar el valor \(p = \frac{dy}{dx}\), con lo cual
nos aparece: \[\frac{dy}{dx} = \pm\sqrt{C_1 y - 1}\]

Ésta es de nuevo una ecuación en variables separadas, que podemos
escribir como: \[(C_1 y - 1)^{-1/2} dy = \pm dx\]

Integrándola: \[2\sqrt{C_1 y - 1} = \pm C_1 x + C_2\]

\begin{tcolorbox}[enhanced jigsaw, breakable, colframe=quarto-callout-important-color-frame, left=2mm, coltitle=black, opacityback=0, colbacktitle=quarto-callout-important-color!10!white, bottomtitle=1mm, title=\textcolor{quarto-callout-important-color}{\faExclamation}\hspace{0.5em}{Solución}, titlerule=0mm, arc=.35mm, bottomrule=.15mm, toptitle=1mm, colback=white, rightrule=.15mm, toprule=.15mm, opacitybacktitle=0.6, leftrule=.75mm]

Con lo que ya hemos encontrado las soluciones de la E. D. de partida:
\[2\sqrt{C_1 y - 1} = \pm C_1 x + C_2\]

\end{tcolorbox}

\chapter{Apartado 12': Ecuaciones Homogéneas Respecto a las
Derivadas}\label{apartado-12-ecuaciones-homoguxe9neas-respecto-a-las-derivadas}

\section{Ecuaciones homogéneas
generalizadas}\label{ecuaciones-homoguxe9neas-generalizadas}

Si la ecuación \(F(x, y, y', \ldots, y^{(n)}) = 0\) es tal que, para
\(\alpha\) y \(m\) fijos, \(F\) cumple:
\[ F(\lambda x, \lambda^m u_0, \lambda^{m-1}u_1, \ldots, \lambda^{m-n}u_n) = \lambda^\alpha F(x, u_0, u_1, \ldots, u_n) \]

vamos a ver que, efectuando un cambio tanto de variable independiente
como de dependiente, esta E. D. se podrá transformar en una del tipo
anterior (es decir, en la que no aparecerá la variable independiente).

\begin{tcolorbox}[enhanced jigsaw, breakable, colframe=quarto-callout-note-color-frame, left=2mm, coltitle=black, opacityback=0, colbacktitle=quarto-callout-note-color!10!white, bottomtitle=1mm, title=\textcolor{quarto-callout-note-color}{\faInfo}\hspace{0.5em}{Definición}, titlerule=0mm, arc=.35mm, bottomrule=.15mm, toptitle=1mm, colback=white, rightrule=.15mm, toprule=.15mm, opacitybacktitle=0.6, leftrule=.75mm]

Una ecuación de estas características suele decirse que es
\textbf{homogénea generalizada} de grado \(\alpha\), en la que cada
factor \(x\) contribuye con grado 1, cada \(y\) con grado \(m\), \(y'\)
con grado \(m-1\), \(y''\) con \(m-2\), etcétera.

\end{tcolorbox}

\section{Cambio de variables}\label{cambio-de-variables}

Tomamos dos nuevas variables \(t\) y \(z\) (\(t\) la independiente y
\(z\) la dependiente) que relacionamos con las originales \(x\) e \(y\)
mediante: \[x = e^t, \quad y = e^{mt}z\]

Con esto, si denotamos \(z' = \frac{dz}{dt}\), tenemos:
\[ \frac{dy}{dx} = \frac{dy/dt}{dx/dt} = \frac{me^{mt}z + e^{mt}z'}{e^t} = e^{(m-1)t}(z' + mz) \]

A partir de aquí: \begin{align*}
\frac{d^2 y}{dx^2} &= \frac{d}{dx}\left( \frac{dy}{dx} \right) = \frac{d}{dt}\left( \frac{dy}{dx} \right)\frac{dt}{dx} \\
&= e^{(m-2)t}(z'' + (2m-1)z' + m(m-1)z)
\end{align*}

y así sucesivamente. En general, por inducción, es claro que:
\[ \frac{d^{(k)}y}{dx^{(k)}} = e^{(m-k)t}g_k(z, z', \ldots, z^{(k)}) \]

\section{Transformación de la
ecuación}\label{transformaciuxf3n-de-la-ecuaciuxf3n-1}

De este modo, sustituyendo en la E. D. que estamos intentando resolver,
obtenemos:
\[ F(e^t, e^{mt}z, e^{(m-1)t}(z' + mz), \ldots, e^{(m-n)t}g_n(z, z', \ldots, z^{(n)})) = 0 \]

extrayendo \(\lambda = e^t\), esto resulta ser:
\[ e^{\alpha t}F(1, z, (z' + mz), \ldots, g_n(z, z', \ldots, z^{(n)})) = 0 \]

Como \(e^{\alpha t}\) no puede anularse, el otro factor tiene que ser
cero, luego hemos transformado la ecuación de partida en una de la
forma: \[ G(z, z', \ldots, z^{(n)}) = 0 \]

en la que no aparece explícitamente la variable independiente \(t\).
Esta ecuación puede reducirse de orden aplicando el proceso descrito en
el apartado anterior.

\begin{tcolorbox}[enhanced jigsaw, breakable, colframe=quarto-callout-tip-color-frame, left=2mm, coltitle=black, opacityback=0, colbacktitle=quarto-callout-tip-color!10!white, bottomtitle=1mm, title=\textcolor{quarto-callout-tip-color}{\faLightbulb}\hspace{0.5em}{RECETA 12'. Si la ecuación \(F(x, y, y', \ldots, y^{(n)}) = 0\)}, titlerule=0mm, arc=.35mm, bottomrule=.15mm, toptitle=1mm, colback=white, rightrule=.15mm, toprule=.15mm, opacitybacktitle=0.6, leftrule=.75mm]

es tal que, para \(\alpha\) y \(m\) fijos, \(F\) cumple:
\[ F(\lambda x, \lambda^m u_0, \lambda^{m-1}u_1, \ldots, \lambda^{m-n}u_n) = \lambda^\alpha F(x, u_0, u_1, \ldots, u_n) \]

haciendo el cambio \(x = e^t\), \(y = e^{mt}z\) la E. D. se transforma
en una de la forma \(G(z, z', \ldots, z^{(n)}) = 0\), a la que se puede
aplicar el método anterior.

\end{tcolorbox}

\subsection*{Ejemplo 12'}\label{ejemplo-12-1}
\addcontentsline{toc}{subsection}{Ejemplo 12'}

\textbf{Resolver} \[ 4x^2 y^3 y'' = x^2 - y^4. \]

\textbf{Análisis de homogeneidad:}

Para que esta ecuación sea homogénea generalizada, cada monomio tiene
que ser del mismo grado. En el miembro de la derecha, esto se consigue
con \(2 = 4m\), es decir, \(m = \frac{1}{2}\), con lo cual \(x^2 - y^4\)
es de grado 2.

El miembro de la izquierda es de grado
\(2 + 3m + (m-2) = 2 + \frac{3}{2} + (\frac{1}{2} - 2) = 2\),
coincidente con el del otro miembro.

Así pues, con \(m = \frac{1}{2}\) la E. D. es homogénea generalizada de
grado 2.

\textbf{Solución:}

(El desarrollo completo está disponible en el documento LaTeX original)

\chapter{Apartado 13: Ecuaciones Homogéneas Respecto a Ciertos
Argumentos}\label{apartado-13-ecuaciones-homoguxe9neas-respecto-a-ciertos-argumentos}

\section{\texorpdfstring{Ecuaciones homogéneas respecto a
\(y, y', \ldots, y^{(n)}\)}{Ecuaciones homogéneas respecto a y, y\textquotesingle, \textbackslash ldots, y\^{}\{(n)\}}}\label{ecuaciones-homoguxe9neas-respecto-a-y-y-ldots-yn}

Si la ecuación \(F(x, y, y', \ldots, y^{(n)}) = 0\) es tal que, para
\(\alpha\) fijo, \(F\) cumple:
\[ F(x, \lambda u_0, \lambda u_1, \ldots, \lambda u_n) = \lambda^\alpha F(x, u_0, u_1, \ldots, u_n) \]

vamos a comprobar cómo, a través de un cambio de variable dependiente,
el orden se puede reducir en uno.

\begin{tcolorbox}[enhanced jigsaw, breakable, colframe=quarto-callout-note-color-frame, left=2mm, coltitle=black, opacityback=0, colbacktitle=quarto-callout-note-color!10!white, bottomtitle=1mm, title=\textcolor{quarto-callout-note-color}{\faInfo}\hspace{0.5em}{Definición}, titlerule=0mm, arc=.35mm, bottomrule=.15mm, toptitle=1mm, colback=white, rightrule=.15mm, toprule=.15mm, opacitybacktitle=0.6, leftrule=.75mm]

La propiedad que caracteriza a la función \(F\) suele expresarse
diciendo que \(F\) es \textbf{homogénea} de grado \(\alpha\) respecto a
los argumentos \(u_0, u_1, \ldots, u_n\).

Con este lenguaje, la ecuación que estamos intentando resolver es
homogénea de grado \(\alpha\) respecto de \(y, y', \ldots, y^{(n)}\).

\end{tcolorbox}

\section{Solución trivial}\label{soluciuxf3n-trivial}

En primer lugar, es claro que, cuando \(\alpha > 0\), la función
constante \(y = 0\) (para la cual \(y' = y'' = \cdots = 0\)) es solución
ya que, efectivamente:
\[F(x, y, y', \ldots, y^{(n)}) = F(x, 0, 0, \ldots, 0) = 0^\alpha F(x, 1, 1, \ldots, 1) = 0\]

\section{Cambio de función}\label{cambio-de-funciuxf3n}

Para hallar las demás soluciones, tomemos una nueva función \(z\) dada
por: \[y' = yz\]

Obviamente, esto es equivalente a decir \(y = \exp(\int z\,dx)\), puesto
que:
\[ \frac{dy}{dx} = yz \Longleftrightarrow \frac{dy}{y} = z\,dx \Longleftrightarrow \log y = \int z\,dx \Longleftrightarrow y = \exp\left( \int z\,dx \right) \]

\subsection*{Derivadas sucesivas}\label{derivadas-sucesivas-1}
\addcontentsline{toc}{subsection}{Derivadas sucesivas}

Si derivamos sucesivamente la expresión \(y' = yz\), obtenemos:
\[y'' = y'z + yz' = y(z^2 + z')\]
\[y''' = y''z + y'z' + yz'' = y(z^3 + 3zz' + z'')\]

y así sucesivamente.

\begin{tcolorbox}[enhanced jigsaw, breakable, colframe=quarto-callout-important-color-frame, left=2mm, coltitle=black, opacityback=0, colbacktitle=quarto-callout-important-color!10!white, bottomtitle=1mm, title=\textcolor{quarto-callout-important-color}{\faExclamation}\hspace{0.5em}{Important}, titlerule=0mm, arc=.35mm, bottomrule=.15mm, toptitle=1mm, colback=white, rightrule=.15mm, toprule=.15mm, opacitybacktitle=0.6, leftrule=.75mm]

Es importante destacar el hecho de que siempre aparece una relación del
tipo \(y^{(k)} = yg_k(z, z', \ldots, z^{(k-1)})\).

\end{tcolorbox}

\section{Transformación de la
ecuación}\label{transformaciuxf3n-de-la-ecuaciuxf3n-2}

Sin más que sustituir, en la E. D. queda:
\[ F(x, y, yz, y(z^2 + z'), \ldots, yg_n(z, z', \ldots, z^{(n-1)})) = 0 \]

de donde, extrayendo \(\lambda = y\):
\[ y^\alpha F(x, 1, z, z^2 + z', \ldots, g_n(z, z', \ldots, z^{(n-1)})) = 0 \]

y por tanto, como estamos suponiendo que \(y\) no es la función nula, el
segundo factor habrá de ser cero. Claramente, esto puede ponerse en la
forma: \[ G(x, z, z', \ldots, z^{(n-1)}) = 0 \]

que es una \textbf{ecuación de orden \(n-1\)}.

Si logramos resolverla, tendremos que sus soluciones serán
\(z = \phi(x, C_1, \ldots, C_{n-1})\), una familia dependiente de
\(n-1\) constantes. Entonces, las soluciones de la E. D. original serán:
\[ y = \exp\left( \int \phi(x, C_1, \ldots, C_{n-1})\,dx \right) \]

lo que, al integrar, introduce la \(n\)-ésima constante. Así, puede
ponerse:
\[ y = C_n \exp\left( \int \phi(x, C_1, \ldots, C_{n-1})\,dx \right) \]

\begin{tcolorbox}[enhanced jigsaw, breakable, colframe=quarto-callout-tip-color-frame, left=2mm, coltitle=black, opacityback=0, colbacktitle=quarto-callout-tip-color!10!white, bottomtitle=1mm, title=\textcolor{quarto-callout-tip-color}{\faLightbulb}\hspace{0.5em}{RECETA 13. Si la ecuación \(F(x, y, y', \ldots, y^{(n)}) = 0\)}, titlerule=0mm, arc=.35mm, bottomrule=.15mm, toptitle=1mm, colback=white, rightrule=.15mm, toprule=.15mm, opacitybacktitle=0.6, leftrule=.75mm]

es tal que, para \(\alpha\) fijo, \(F\) cumple:
\[ F(x, \lambda u_0, \lambda u_1, \ldots, \lambda u_n) = \lambda^\alpha F(x, u_0, u_1, \ldots, u_n) \]

entonces el cambio de función dado por \(y' = yz\) (es decir,
\(y = \exp(\int z\,dx)\)) hace que el orden se reduzca en uno.

\end{tcolorbox}

\subsection*{Ejemplo 13}\label{ejemplo-13}
\addcontentsline{toc}{subsection}{Ejemplo 13}

\textbf{Resolver} \[ 3x^2((y')^2 - yy'') = y^2. \]

\textbf{Solución:}

Esta ecuación es homogénea de grado 2 respecto de \(y, y', y''\) pues,
al sustituir \(y, y', y''\) por \(\lambda y, \lambda y', \lambda y''\),
la ecuación queda multiplicada por \(\lambda^2\).

Para resolverla, hacemos el cambio \(y\) por \(z\) dado por
\(y = \exp(\int z(x)\,dx)\), con lo cual:
\[ y' = z\exp\left( \int z\,dx \right), \quad y'' = (z' + z^2)\exp\left( \int z\,dx \right) \]

Sustituyendo en la E. D. queda:
\[ 3x^2\left[ z^2\exp\left( 2 \int z\,dx \right) - (z' + z^2)\exp\left( 2 \int z\,dx \right) \right] = \exp\left( 2 \int z\,dx \right) \]

de donde, simplificando \(\exp(2 \int z\,dx)\), se sigue
\(-3x^2 z' = 1\), ecuación en variables separadas.

Tras ponerla como \(dz = -\frac{1}{3}x^{-2}\,dx\) la integramos, con lo
que \(z = \frac{1}{3}x^{-1} + C_1\).

Por último: \begin{align*}
y &= \exp\left( \int z(x)\,dx \right) = \exp\left( \int \left( \frac{1}{3}x^{-1} + C_1 \right) dx \right) \\
&= \exp\left( \frac{1}{3}\log x + C_1 x + C_2 \right) = x^{1/3}\exp(C_1 x + C_2)
\end{align*}

\begin{tcolorbox}[enhanced jigsaw, breakable, colframe=quarto-callout-important-color-frame, left=2mm, coltitle=black, opacityback=0, colbacktitle=quarto-callout-important-color!10!white, bottomtitle=1mm, title=\textcolor{quarto-callout-important-color}{\faExclamation}\hspace{0.5em}{Solución}, titlerule=0mm, arc=.35mm, bottomrule=.15mm, toptitle=1mm, colback=white, rightrule=.15mm, toprule=.15mm, opacitybacktitle=0.6, leftrule=.75mm]

La solución de la ecuación original es, sin más que cambiar la notación
de las constantes, la familia de funciones: \[y = K_1 x^{1/3}e^{K_2 x}\]

\end{tcolorbox}

\bookmarksetup{startatroot}

\chapter{Bibliografía}\label{bibliografuxeda}

\section{Referencias}\label{referencias}

A continuación se presenta la bibliografía recomendada para profundizar
en el estudio de ecuaciones diferenciales ordinarias:

\begin{enumerate}
\def\labelenumi{\arabic{enumi}.}
\item
  \textbf{F. Ayres}, \emph{Ecuaciones diferenciales}, Col. Schaum,
  McGraw-Hill, México, 1988.
\item
  \textbf{M. Braun}, \emph{Ecuaciones diferenciales y sus aplicaciones},
  Grupo Editorial Iberoamérica, México, 1990.
\item
  \textbf{W. E. Boyce y R. C. DiPrima}, \emph{Ecuaciones diferenciales y
  problemas con valores en la frontera}, Limusa, México, 1992 (3.ª
  edición).
\item
  \textbf{E. A. Coddington}, \emph{Introducción a las ecuaciones
  diferenciales ordinarias}, CECSA, México, 1980.
\item
  \textbf{M. W. Hirsch y S. Smale}, \emph{Ecuaciones diferenciales,
  sistemas dinámicos y álgebra lineal}, Alianza Universidad Textos,
  Madrid, 1983.
\item
  \textbf{S. G. Krantz}, \emph{Differential Equations. Theory, Technique
  and Practice}, CRC Press, Boca Raton, 2015.
\item
  \textbf{L. Pontriaguine}, \emph{Ecuaciones diferenciales ordinarias},
  Mir, Moscú, 1969.
\item
  \textbf{G. F. Simmons}, \emph{Ecuaciones diferenciales con
  aplicaciones y notas históricas}, McGraw-Hill, México, 1977.
\item
  \textbf{J. Stewart}, \emph{Cálculo, conceptos y contextos}, Thomson,
  México, 1999.
\end{enumerate}

\begin{tcolorbox}[enhanced jigsaw, breakable, colframe=quarto-callout-tip-color-frame, left=2mm, coltitle=black, opacityback=0, colbacktitle=quarto-callout-tip-color!10!white, bottomtitle=1mm, title=\textcolor{quarto-callout-tip-color}{\faLightbulb}\hspace{0.5em}{Recursos adicionales}, titlerule=0mm, arc=.35mm, bottomrule=.15mm, toptitle=1mm, colback=white, rightrule=.15mm, toprule=.15mm, opacitybacktitle=0.6, leftrule=.75mm]

Para consulta en línea, se recomienda: -
\href{http://mathworld.wolfram.com/}{Wolfram MathWorld - Differential
Equations} - \href{https://ocw.mit.edu/}{MIT OpenCourseWare -
Differential Equations}

\end{tcolorbox}

\bookmarksetup{startatroot}

\chapter{Apéndice: Fórmulas de
Integración}\label{apuxe9ndice-fuxf3rmulas-de-integraciuxf3n}

\section{Reglas principales de
integración}\label{reglas-principales-de-integraciuxf3n}

\subsection*{1. Integral indefinida}\label{integral-indefinida}
\addcontentsline{toc}{subsection}{1. Integral indefinida}

Si \(F'(x) = f(x)\), entonces: \[\int f(x) \, dx = F(x) + C\]

\subsection*{2. Integrales inmediatas}\label{integrales-inmediatas}
\addcontentsline{toc}{subsection}{2. Integrales inmediatas}

\section{Potencias y logaritmo}

\[\int x^n \, dx = \frac{x^{n+1}}{n+1} + C \quad (n \neq -1)\]

\[\int \frac{1}{x} \, dx = \ln|x| + C\]

\section{Exponencial}

\[\int e^x \, dx = e^x + C\]

\[\int a^x \, dx = \frac{a^x}{\ln a} + C \quad (a > 0, a \neq 1)\]

\section{Trigonométricas}

\[\int \operatorname{sen} x \, dx = -\cos x + C\]

\[\int \cos x \, dx = \operatorname{sen} x + C\]

\[\int \operatorname{tg} x \, dx = -\ln|\cos x| + C\]

\[\int \operatorname{cotg} x \, dx = \ln|\operatorname{sen} x| + C\]

\[\int \sec x \, dx = \ln|\sec x + \operatorname{tg} x| + C\]

\[\int \csc x \, dx = \ln|\csc x - \operatorname{cotg} x| + C\]

\section{Trigonométricas inversas}

\[\int \frac{1}{\sqrt{1-x^2}} \, dx = \operatorname{arcsen} x + C\]

\[\int \frac{1}{1+x^2} \, dx = \operatorname{arctg} x + C\]

\[\int \frac{1}{x\sqrt{x^2-1}} \, dx = \operatorname{arcsec} |x| + C\]

\subsection*{3. Método de
sustitución}\label{muxe9todo-de-sustituciuxf3n}
\addcontentsline{toc}{subsection}{3. Método de sustitución}

Si \(u = g(x)\), entonces: \[\int f(g(x))g'(x) \, dx = \int f(u) \, du\]

\subsection*{4. Integración por partes}\label{integraciuxf3n-por-partes}
\addcontentsline{toc}{subsection}{4. Integración por partes}

\[\int u \, dv = uv - \int v \, du\]

o equivalentemente:
\[\int u(x)v'(x) \, dx = u(x)v(x) - \int u'(x)v(x) \, dx\]

\subsection*{5. Fracciones racionales}\label{fracciones-racionales}
\addcontentsline{toc}{subsection}{5. Fracciones racionales}

Para integrar \(\frac{P(x)}{Q(x)}\) donde \(P\) y \(Q\) son polinomios:

\begin{enumerate}
\def\labelenumi{\arabic{enumi}.}
\tightlist
\item
  Si grado de \(P \geq\) grado de \(Q\): dividir primero
\item
  Descomponer en fracciones simples
\item
  Integrar cada fracción simple
\end{enumerate}

\begin{tcolorbox}[enhanced jigsaw, breakable, colframe=quarto-callout-tip-color-frame, left=2mm, coltitle=black, opacityback=0, colbacktitle=quarto-callout-tip-color!10!white, bottomtitle=1mm, title=\textcolor{quarto-callout-tip-color}{\faLightbulb}\hspace{0.5em}{Descomposición en fracciones simples}, titlerule=0mm, arc=.35mm, bottomrule=.15mm, toptitle=1mm, colback=white, rightrule=.15mm, toprule=.15mm, opacitybacktitle=0.6, leftrule=.75mm]

\begin{itemize}
\item
  Para \((x-a)^n\): términos
  \(\frac{A_1}{x-a} + \frac{A_2}{(x-a)^2} + \cdots + \frac{A_n}{(x-a)^n}\)
\item
  Para \(x^2 + bx + c\) irreducible: término
  \(\frac{Ax + B}{x^2 + bx + c}\)
\end{itemize}

\end{tcolorbox}

\subsection*{Tabla de integrales
útiles}\label{tabla-de-integrales-uxfatiles}
\addcontentsline{toc}{subsection}{Tabla de integrales útiles}

\begin{longtable}[]{@{}
  >{\raggedright\arraybackslash}p{(\columnwidth - 2\tabcolsep) * \real{0.4762}}
  >{\raggedright\arraybackslash}p{(\columnwidth - 2\tabcolsep) * \real{0.5238}}@{}}
\toprule\noalign{}
\begin{minipage}[b]{\linewidth}\raggedright
Integral
\end{minipage} & \begin{minipage}[b]{\linewidth}\raggedright
Resultado
\end{minipage} \\
\midrule\noalign{}
\endhead
\bottomrule\noalign{}
\endlastfoot
\(\displaystyle\int \frac{1}{a^2+x^2} \, dx\) &
\(\displaystyle\frac{1}{a}\operatorname{arctg}\frac{x}{a} + C\) \\
\(\displaystyle\int \frac{1}{\sqrt{a^2-x^2}} \, dx\) &
\(\displaystyle\operatorname{arcsen}\frac{x}{a} + C\) \\
\(\displaystyle\int \frac{1}{a^2-x^2} \, dx\) &
\(\displaystyle\frac{1}{2a}\ln\left|\frac{a+x}{a-x}\right| + C\) \\
\(\displaystyle\int \frac{1}{\sqrt{x^2 \pm a^2}} \, dx\) &
\(\displaystyle\ln|x + \sqrt{x^2 \pm a^2}| + C\) \\
\(\displaystyle\int \sqrt{a^2-x^2} \, dx\) &
\(\displaystyle\frac{x}{2}\sqrt{a^2-x^2} + \frac{a^2}{2}\operatorname{arcsen}\frac{x}{a} + C\) \\
\end{longtable}

\begin{tcolorbox}[enhanced jigsaw, breakable, colframe=quarto-callout-note-color-frame, left=2mm, coltitle=black, opacityback=0, colbacktitle=quarto-callout-note-color!10!white, bottomtitle=1mm, title=\textcolor{quarto-callout-note-color}{\faInfo}\hspace{0.5em}{Recursos para más fórmulas}, titlerule=0mm, arc=.35mm, bottomrule=.15mm, toptitle=1mm, colback=white, rightrule=.15mm, toprule=.15mm, opacitybacktitle=0.6, leftrule=.75mm]

Para una lista más completa de fórmulas de integración, consultar:

\begin{itemize}
\tightlist
\item
  Tablas de integrales de Gradshteyn y Ryzhik
\item
  \href{http://integrals.wolfram.com/}{Wolfram Integrator}
\item
  \href{https://www.symbolab.com/}{Symbolab Calculator}
\end{itemize}

\end{tcolorbox}


\backmatter

\end{document}
